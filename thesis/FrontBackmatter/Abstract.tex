%*******************************************************
% Abstract
%*******************************************************
%\renewcommand{\abstractname}{Abstract}
\pdfbookmark[1]{Abstract}{Abstract}
\begingroup
\let\clearpage\relax
\let\cleardoublepage\relax
\let\cleardoublepage\relax

\chapter*{Abstract}

Many computations running on high-performing systems does not make use
of the performance available. This problem is related to computations
failing to achieve strong scaling.

Copernicus is a system for execution of large-scale sampling tasks in
high-performance environments. It aims to achieve strong scaling,
regardless of underlying architecture. The system was originally
developed to run large scale bio-molecular simulations. However,
lacking an intuitive way of describing computational projects, the
developers felt a need for an user-friendly text-based input for
Copernicus.

This master's thesis describes a design and implementation of a
domain-specific language to meet the need of a suitable input
description for Copernicus. The language design is simple but still
manages to capture the abstract model of how a computational project
is executed. The language is strongly typed and inspired by elements
from both functional programming and data-flow languages. In the end,
Rheos proves to be a powerful descriptive domain-specific language.

\endgroup			

\vfill
