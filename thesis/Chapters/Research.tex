\chapter{Research}
This chapter will give a brief exaplanation on how the project was
executed, what stages .


%\section{Project overview}
%The project contained three main stages: research, design and
%implementation. The first plan was to sequentially work through them
%one by one. In practice a more parallel approach was taken, where
%designing the DSL started before the research ended and then the
%implementation and design stages continued through the entire project.


\section{Programming Paradigms}
We researched programming paradigms and programming languages similar
to what we wanted to build.

\subsection{Dataflow Programming}
Dataflow programming is a paradigm that models programs as a directed
graph of the data flowing between operations. It focuses on how
components of the program \emph{connects} instead of, as in imperative
programming, \emph{how they happen}. \highlight{Expand on this.}

\subsection{Flow-based Programming}
In flow-based programming, applications are defined as networks of
``black box'' processes. Data is exchanged across predefined
connections via message passing. \highlight{Expand on this.}

\subsection{Reactive Programming}
Oriented around data flows and the propagation of change. THe
underlying execution model will automatically propagate changes
through the data flow. \highlight{Expand on this.}


\section{Requirements \& Specification}

