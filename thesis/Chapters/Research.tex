\chapter{Research}\label{chap:research}
This chapter presents some of the research that was done on
Copernicus, different programming paradigms suitable for the problem
domain, and existing programming languages appurtenant to those
paradigms. Maybe some more stuff to...

%\section{Project overview}
%The project contained three main stages: research, design and
%implementation. The first plan was to sequentially work through them
%one by one. In practice a more parallel approach was taken, where
%designing the DSL started before the research ended and then the
%implementation and design stages continued through the entire project.


\section{Requirements \& Specification}


\section{Programming Paradigms}
Different problem domains call for different programming
paradigms. The execution model of \emph{Copernicus} can be thought of
as a flow network, which makes some paradigms more interesting, and
the domain-specific language should be made to reflect that fact.


\subsection{Dataflow Programming}
Dataflow programming is a paradigm which has an execution model where
a program is represented as directed graph. The data flows between
operations along the arcs. Directed arcs represent dependencies
between instructions. Arcs that flow toward a node are called
\emph{inputs}, while arcs flowing away from a node are called
\emph{outputs} \citep{johnston:2004}. The model focuses on how
components of the program \emph{connects} in contrast to the classical
von Neuman model, which focuses on \emph{how they happen}.


\subsection{Flow-based Programming}
In flow-based programming (FBP), applications are defined as networks
of ``black box'' processes. Data is exchanged across predefined
connections via message passing. The black box processes can be
reconnected in different ways to form new applications while their
internals remain unchanged, thus making FBP a
\emph{component-oriented} approach
\citep{morrison:2010,morrison:online}.


\subsection{Reactive Programming}
Reactive programming is oriented around data flows and the propagation
of change. The key ideas are notions of \emph{behaviours} and
\emph{events}, where behaviours are reactive values that varies over
time, while events are time-ordered sequences of discrete-time event
occurences \citep{wan:2000}. The underlying execution model will
automatically propagate changes through the data flow.


\section{Programming Languages}



