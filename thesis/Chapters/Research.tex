\chapter{Research}\label{chap:research}
This chapter presents some of the research that was done on
Copernicus, different programming paradigms suitable for the problem
domain, and existing programming languages belonging to those
paradigms.


\section{Programming Paradigms}
Different problem domains call for different programming
paradigms. The execution model of \emph{Copernicus} can be thought of
as a flow network, which makes some paradigms more interesting, and
the domain-specific language should be made to reflect that fact. The
data-flow programming model contrasts the classical control flow model
implemented in languages such as C.

Applications written in \eg~C have inherent limitations when run in
parallel environments, because of the top-down sequential programming
approach. The data-flow model consists of nodes connected to
each-other to express the logical execution flow, and it can easily be
used to express parallelism.


\subsection{Data-flow Programming}
The origin of data-flow languages is related the ever increasing need
for parallelism in today's applications. Data-flow programming is a
paradigm which has an execution model where a program is represented
as directed graph. The data flows between operations along the
arcs. Directed arcs represent dependencies between instructions. Arcs
that flow toward a node are called \emph{inputs}, while arcs flowing
away from a node are called \emph{outputs} \citep{johnston:2004}. The
model focuses on how components of the program \emph{connects} in
contrast to the classical Von Neuman model, which focuses on \emph{how
  they happen}.


\subsection{Flow-Based Programming}
In flow-based programming (FBP), applications are defined as networks
of ``black box'' processes. Data is exchanged across predefined
connections via message passing. The black box processes can be
reconnected in different ways to form new applications while their
internals remain unchanged, thus making FBP a
\emph{component-oriented} approach
\citep{morrison:2010,morrison:online}.


\subsection{Reactive Programming}
Reactive programming is oriented around data flows and the propagation
of change. The key ideas are notions of \emph{behaviors} and
\emph{events}, where behaviors are reactive values that varies over
time, while events are time-ordered sequences of discrete-time event
occurrences \citep{wan:2000}. The underlying execution model will
automatically propagate changes through the data flow.


\section{Programming Languages}
There are some existing implementations of FBP out there. However,
these implementations mainly consist of language extensions or
libraries for general-purpose languages. Some of these language
extensions includes \emph{THREADS, JavaFBP, C\#FBP} and
\emph{DrawFBP} \citep{morrison:2010}.



%\highlight{A little too short. How do they integrate with a
%general-purpose language?}
