\chapter{Conclusions}

\section{The Problem At Hand}
As the main objective was to find a solution to Copernicus problem of
describing its projects, Rheos is an answer to this projects main
goal. It is possible to describe computational project for Copernicus
and Rheos supports all current aspects of the Copernicus system.

As required Rheos is a text-based way of describing project networks
in the form of a descriptive domain-specific language. Rheos supports
arbitrary plug-ins, since you can describe how and what Copernicus
should execute. As long as Copernicus as access to that executable any
kind of plug-in is possible to use in Rheos.

Rheos is simple because of its limited capabilities. The point of the
DSL is to describe the project and not to execute or evaluate any
code, but it is still a powerful tool for the users.

\section{State of the Implementation}
While the Rheos never got to a state where it could be integrated with
Copernicus, and still has some implementation steps that needs to be
addressed, the project as a whole got very far. Over the course of the
last months, Rheos has evolved from just being a replacement for the
XML-description of Copernicus' input to a much more powerful
language. Much time was spent on design and re-design of the
language. New features was continuously added to the language which
made it hard to for the project to reach the point where Rheos was
integrated with Copernicus. It is important to have a powerful tool so
that Rheos is useful to the users, and that is why adding features was
prioritized. The integration with Copernicus was left to its
developers.

\section{What Rheos Became}

\subsection{A New Approach}
Rheos is a completely new solution for input to systems like
Copernicus. The language is designed specifically for Copernicus and
its users. The idea is for users to fast and easily understand and use
Rheos as a minor step when running computations on a Copernicus
system. This is not the purpose for other distributed processing
systems. The need for a solution like Rheos has probably not been
encountered before Copernicus was developed, because of reasons like
this.

Packages like Hadoop has the intention of building application which
interfaces with a distributed processing system, which is not the
desired approach for Rheos and Copernicus.

\subsection{Programing Style}
Using Rheos for describing computational projects is a sort of
metaprogramming. Rheos describes and manipulates how a Copernicus
system should evaluate a project. It is not the intention for users
to think of it this way, but to just to supply a problem and get an
answer. Rheos meet this intention since the users describes workflows
and send them to a Copernicus system.

Rheos has a flow-based nature, which is necessary to correctly
describe projects in Copernicus. There are also some functional
aspects to the language, which makes describing projects easier and
provides a good overview of networks as the code will be dense. The
user can choose to describe its project as both a workflow and with
some functional ways of initializing components in the network.


\subsection{Powerful Description Language}
The DSL is stronger than just a set of macros. With Rheos it is
possible to express generic parallel algorithms which is not the case
with macros for just adding components and connections. Pre-defined
code would be easier for users to understand and use with the
definition and the type-system available. As the code describes
networks it should be easier to understand what the network actually
look like.

The strong type-system makes Rheos a much safer way of building
networks compared to the old input system where computations would
fail on runtime when Copernicus actually would build the project
networks. The polymorphism makes the type-system powerful for a
description language. It makes using pre-defined code much easier for
users, where they can for example map functions with any kind of
inputs.

With Rheos, Copernicus can now be used by non-developer users and they
can still use it as a powerful tool. This is very important since many
of the intended users are included in this category, and was the
inspiration for the entire project. It is still possible to use Rhoes
in a complicated way, which developers whould like. Rheos has in the
end met the needs of Copernicus and made it easy for users to describe
complex or big computations in a simple way.
