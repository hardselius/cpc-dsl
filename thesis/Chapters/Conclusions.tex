\chapter{Conclusions}

\section{The Problem At Hand}
As the main objective was to find a solution to Copernicus problem of
describing its projects, Rheos is an answer to this projects main
goal. It is possible to describe computational project for Copernicus
and Rheos supports all current aspects of the Copernicus system.

As required Rheos is a text-based way of describing project networks
in the form of a descriptive domain-specific language. Rheos supports
arbitrary plug-ins, since you can describe how and what Copernicus
should execute. As long as Copernicus as access to that executable any
kind of plug-in is possible to use in Rheos.

Rheos is simple because of its limited capabilities. The point of the
DSL is to describe the project and not to execute or evaluate any
code, but it is still a powerful tool for the users.


\section{What Rheos Became}

\subsection{A New Approach}
Completely new

differences from other languages or packages

\subsection{Programing Style}
meta programming

functional data-flow programming

\subsection{Powerful Description Language}
more than just macros

descriptive net

Express generic parallel algorithms

\subsection{Type-system}
strongly typed

Powerful type-system


\section{State of the Implementation}
While the Rheos never got to a state where it could be integrated with
Copernicus, and still has some implementation steps that needs to be
addressed, the project as a whole got very far. Over the course of the
last months, Rheos has evolved from just being a replacement for the
XML-description of Copernicus' input to a much more powerful
language. Alot of time was spent on design and re-design, which has
both advantages and disadvantages.
