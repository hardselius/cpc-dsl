\chapter{Conclusions}

\section{State of the Implementation}
While the Rheos never got to a state where it could be integrated with
Copernicus, and still has some implementation steps that needs to be
addressed, the project as a whole got very far. Over the course of the
last months, heos has evolved from just being a replacement for the
XML-description of Copernicus' input to a much more powerful
language. Alot of time was spent on design and re-design, which has
both advantages and disadvantages.


\section{What Rheos Became}

\subsection{A New Approach}

\subsection{Programing Style}

\subsection{Powerful Description Language}

\subsection{Type-system}


\section{Blah}
\begin{itemize}
\item more to be done before use, but close enough
\item alot of time spent on design/re-design, which led to that we did
  not come as far as we thought. We expected to have running version
  intergtrated with Copernicus
\item usful reuslt
\item we are happy and the developers of Copernicus seems to be happy
  with the result
\item differences from other languages
\item descriptive net
\item strongly typed
\item data-flow functional components
\item meta programming
\item more than just macros
\end{itemize}

\paragraph{PROBLEM STATEMENT ANSWERS}
\begin{itemize}
\item Rheos is a solution to the problem Copernicus had
\item It is a text-based way to descrube projects for Copernicus
\item Rheos is simple due to its limited descriptive nature (no
  execution/evaluation).
\end{itemize}

\begin{enumerate}
\item Completely new
\item Functional \& flow-based programming
\item Express generic parallell algorithms
\item Powerful type-system
\end{enumerate}
