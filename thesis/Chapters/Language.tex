\chapter{Language}\label{chap:language}
This chapter describes the domain-specific language, as well as some
rationale behind the design and future work.

For a complete description of the grammar in Backus-Naur Form (BNF),
see \autoref{sec:bnf}. The implementation details will be described
further in \autoref{chap:implementation}.

\section{Design}
Designing the DSL was a process which continued through the entire
project. As there are no comparable packages and no text-based
solution to similar problems, the DSL had no real starting base. The
initial inspiration came from the reasearch on programming praradigms
and graphical implementations related to network based programming.

Inspiration from well known programming languages was included for the
DSL to be simple and intuitive to the common user. Both functional and
imperative languages were considered when developing the design.

The most important steps in this process was a continuing discussion
with the developers of Copernicus. It was important to have a DSL
which they were satisfied with, but also to get input on what design
choices to make. The developers perspective was important for the DSL
to reflect realistic scenarios and to get a better collective view of
the different solutions. At each meeting the developers was presented
with a draft of the latest version of the DSL.

\section{General style and features}
The DSL is a descriptive language which has four types of top-level
syntactic definitions: atoms, networks, imports and new types. Atoms
and networks are components which are connected to build project
networks in Copernicus. The four top-level definitions are described
more in detail in the following sections. All the top-level
definitions needs to be defined before used in the current version of
the DSL.

Atoms and networks both have a set of external inputs and outputs. An
input/output does not need to have a connection since atoms/networks
only evaluates once it has all non-optional inputs has received a new
value. It is possible to add new atoms/networks and connections in the
Copernicus system, but as there are currently no interactive
implementation of the dsl such actions cannot be done with this
version of the solution. An input/output can be assigned constant
values if they are of primitive types. These features are explained in
more detail in this chapter.


\section{Modules}
Modules have no real function in the current implementation of the
language. A module is basically a code file. Importing a file is
importing a module. This is how to import the file
``path/to/file.cod'' from subdirectories of the current files path:

\begin{verbatim}
import path.to.file
\end{verbatim}

All code from the imported module is simply added where the import
statement is located in the code. This way the code from the imported
file will be used as it would have been written in the current
file. This system is not designed or developed to have modules and
packages as wrappers for code, but rather to sort code in different
files. 

Currently it is only possible to import files from
subdirectories. There are no way to import from other paths or any
form of a standard library. The discussion of a more advanced and
useful import system revealed that the developers had some different
suggestions but had not decided how such a system should work with
client server setup in Copernicus.

\section{Typing}
In the language all connections between executables are typed. Their
types are set when inputs and outputs are defined. Atoms and even
networks have types according to their external inputs and outputs.

The typechecker does not typecheck executables. Instead the system
relies on the correct definition when wrapping them in the language.

\subsection{Primitive Types}
The first primitive is integers, which are represented by
\verb#int#. Integers cannot be assigned floating points (implicitly
turn floating point to integer) but has to be assigned
integers. Floating points are represented as \verb#float#, and as
integers floating points cannot be assigned anything other than
floating points. This means that \verb#1.0# is not the same as
\verb#1# in the language and will cause a type error if assigned to a
connection with the wrong type.

Strings are represented by \verb#string#. A constant strings is
written between the two quotation symbols. The last primitive type is
files, which are described as \verb#file#. Files are paths written
like strings, but are not checked that it is a path to a file.

\subsection{Compound Types}\label{sec:compound}
There are two compound types in the language. The first one are
arrays, which are ordered lists of a certain type. An integer array is
written like \verb#int[]#, where each pair of brackets represent one
dimension. It is possible to have any number of dimensions of
arrays. Accessing the fourth element of a variable \verb#x# which is
typed as the example integer array, is written \verb#x[4]#. The
resulting type of accessing elements is the same type with one less
dimension. In the example that would be an integer (\verb#int#).

The second compound type is records. They are orderd sets of elements
which are assigned names. This make it possible to access elements in
a record by both its index and name. Accessing the fourth variable
\verb#e# of a record \verb#x# using its name would be written as
\verb#x.e#, and to access \verb#x# by its index would be written as
\verb#x(4)#. The elements can have any kind of type and does not have
to have the same type as the other elements in the record. A new type
has to be defined to represent a record.

\subsection{New Types}
Defining new types are the way records are described in the
language. A new type needs a name and which types of elements it
contains. Each element needs a type and a unique name in the
record. New types can only be defined outside networks and atoms.

The syntax for defining a new type begins with the keyword
\verb#type#. Consider a record called \verb#setting# containing a file
called \verb#f# and a string called \verb#name#. The following code
represents this record \verb#setting#.

\begin{verbatim}
type setting
    ( file  : f
    , float : name )
\end{verbatim}

The file has in this case index zero and the file has index one. They
can be access with both name and index as described above.

\subsection{Atom \& Network Type}\label{sec:atomnet}
Atoms and networks are components used to build networks with. They
work like black boxes in data-flow networks. As components are able to
have any number of inputs and outputs their types are a special case
of a record. Such records contain three elements: inputs, outputs,
meta parameters. The inputs and outputs are records aswell, in which
every input and output parameter are elements. The meta parameter is a
record which should contain types and is more explained in
\autoref{sec:meta}.

A component with an array of type \verb#setting# called
\verb#settings# and an integer called \verb#length# as inputs and a
file called \verb#output# as outputs, is described by the following
syntax.

\begin{verbatim}
in ( setting[] settings
   , int       length  )
out ( file output )
\end{verbatim}

The full header syntax for networks and atoms is described in
\autoref{sec:net} and \autoref{sec:atom}.

Since both the type of \verb#in# and the type of functions are
records, there are numerous ways of refering to an input
parameter. Refering to the a parameter \verb#length# when a function
of the same type as the above has been instantiated as \verb#func# may
look like \verb#func.in.length# or \verb#func(0)(1)# which both are
the same thing.

\subsection{The 'network' Type}
Besides the syntactic objects \verb#network#, which have types
described in the previous section, there is a type \verb#network#. The
type \verb#network# is mainly used with controllers
(\autoref{sec:control}). The type \verb#network# refers to a record
containing a set of instantiated components and a set of connections.

\subsection{Meta Types}\label{sec:meta}
Components can take types as inputs, called meta types, to be able to
define generic components in networks. The parameters are defined with
a variable name and a type group. There are currently three different
type groups: \verb#func#, \verb#list#, and \verb#type#. The group is a
constraint on the type parameter which accepts only certain types. The
group \verb#type# means the parameter only accepts primitive types,
\verb#list# accepts compound type (records and arrays), and
\verb#func# accepts record types which at least have an element
\verb#in# and an element \verb#out#. This way one can define new types
which can be used to describe types of components.

The type parameters are defined and as the following example:

\begin{verbatim}
< func f , list l , type t >
\end{verbatim}

\begin{verbatim}
in ( f.in i
   , l*   inputs
out ( t[] outputs )
\end{verbatim}

\verb#i# will be of a record type since \verb#in# is a record. The
symbol \verb#*# removes one dimension from the array \verb#l#. The
output \verb#outputs# will be a one dimensional array where the type
is given by the parameter \verb#t#.

%\subsection{Type System}
%what is checked...

%parameters to atoms and networks are checked

%metatypes are checked

\section{Instance Names}
Instance names refer to an instantiated Components. They are
instantiated inside networks, where they can be connected to other
instance names or external inputs and outputs. How to instantiate an
instance name is described in \autoref{sec:stmt}.

Instance names contains letters, numbers, and underscores, but they
have to start with a lower case letter.

\section{Atoms}\label{sec:atom}
An atom is a wrapper for executable code, python scripts and
functions. In the language atoms are components just like networks and
the information of what to execute and how is hidden inside the
lanugage for a more intuitive way of building project
networks. Executables should be wrapped and added to an appropriate
library so users do not have to conern themselves with external
(outside the language) project design.

An atom has a header and an option part. The header contains the name
of the atom, the type parameter definition, the type signature of the
outputs and inputs, and what type of executable the atom uses. There
are three different types of executables in the current version:
\verb#python#, \verb#python-extended#, and
\verb#external#. \verb#python# means that the atom calls built-in
functions of python, \verb#python-extended# means the atom calls
python scripts, and \verb#external# calls binary executables.

The following code is the header for an atom \verb#someatom# which
calls a binary executable and has the type definition used in previous
sections (note that in this case the type parameters are not used but
are there to give a full description of a header).

\begin{verbatim}
atom external someatom < func f , list l , type t >
  in  ( setting[] settings
      , int       length  )
  out ( file output )
\end{verbatim}

The option part is a list of options and values. The options are the
information on what and how to execute for Copernicus. The values are
relative to Copernicus and are not a part of the language, which is
why users should not have to write atoms themselves but import and use
pre-defined atoms.

The following code is the definition of an atom \verb#add# which uses
a built-in python function to add two floating points together.

\begin{verbatim}
atom python add
  in  ( float a
      , float b )
  out ( float o )
  options ( fuction : builtin.float.add
          , import  : builtin.float )
\end{verbatim}

\section{Networks}\label{sec:net}
A network is a description of what to instantiate and how the
components and external input/outputs are connected inside the
network. Networks can be components the same way as atoms, which makes
them subnetworks when instantiated in other networks. Each network has
its own scope of type variables and instance names so the external
inputs/outputs is needed to make connections to an external component.

The header of a network differs from atoms headers on two points. The
key word \verb#atom# is replaced with \verb#network# and a network
does not have an executable type. Writing a network \verb#somenet#
with the same parameters and type signature as \verb#someatom# looks
like this:

\begin{verbatim}
network somenet < func f , list l , type t >
  in  ( setting[] settings
      , int       length  )
  out ( file output )
\end{verbatim}

The second part of a network is its network body which is a list of
statements seperated by new lines. These statements are the
description of the network, and they are described in detail in the
next section.

\section{Network Description Statements} \label{sec:stmt}
There are three types of statements: assignment, connections, and a
controller statement. With these statements a network can be described
inside a network body. It is not possible to build networks outside a
network definition.

\subsection{Assignment}
Assignment statements are instantiations of components which are
assigned to instance names. An assignement needs an expression with
information of what component and with which meta types and inputs is
going to be instantiated. All the type arguments needs to be given
when instantiated, but as it is not mandatory to connect something to
an input the input arguments can be left empty. The arguments are
connected to their respective input in the order they are defined in
the component.

The type arguments are listed between the symbols \verb#<# and
\verb#>#. They have to be defined on instantiation and are type
references as types are referenced in \autoref{sec:meta} about
meta types. The following example is an instantiation of the network
\verb#somenet# without any input arguments, hence the empty list
\verb#()#. The first type argument is the type of the atom \verb#add#
which is a record, the second type argument is the record of inputs
for \verb#add#, and the last type argument is the primitive type
\verb#float#.

\begin{verbatim}
var = somenet < add , add.in , float> ()
\end{verbatim}

The input arguments are listed between the symbols \verb#(# and
\verb#)#. An input argument can be a reference to a primitive, record
or an array. The references are the same thing as accessing elements
of compound types as in \autoref{sec:compound}. It is also possible to
assign constant values to inputs by.

The following code line instantiates an atom \verb#add# and assigns it
to an instance name \verb#var#. The first input of \verb#add# is
assigned the constant value \verb#1.1# and the second input is
connected to the current networks input called \verb#fp#.

\begin{verbatim}
var = add ( 1.1 , in.fp )
\end{verbatim}

As mentioned, it is not necessary to supply all the input with values
or connections so it is possible to remove \verb#, in.fp# and the
instantiation would still work (where the first input would still be
assigned the constant value).

It is possible to instantiate a component as an input argument
expression, where a specific output is connected to the respective
input. In the following example an atom \verb#mul# is instantiated and
its output \verb#o# is connected to the second input of the atom
\verb#add#. The atom \verb#mul# is instantiated with its arguments
inside parenthases and \verb#.out.o# refers to the output which should
be connected to the second input of \verb#add#.

\begin{verbatim}
var = add ( 1.1 , (mul ( 2.0 , in.fp )).out.o)
\end{verbatim}

\subsection{Connection}
Connection statements connects two inputs/outputs together. The
inputs/outputs are references like the arguments in an assignment
statement. The following code line connects the the output variable
\verb#o# of the instantiated instance name \verb#var# to the external
output \verb#o# of the network.

\begin{verbatim}
out.o <- var.out.o
\end{verbatim}

It is also possible to assign constant values to connections which the
following code line is an example of.

\begin{verbatim}
out.o <- 4.3
\end{verbatim}

\subsection{Controllers} \label{sec:control}
When a component is set to be a controller, it is given permission to
instatiate components and add connections within the current
network. An example of when this could be useful is mapping a function
(atom) over an array. The controller would build an instantiation of
the function for each element in the input array and connect it the
the appropriate element in an output array. For users to create their
own controllers, they would need to know the internals of Copernicus
and wrap an executable in an atom.

The following coode instantiates \verb#somecontroller# and supplies an
array of type \verb#setting#, and then sets the input
\verb#var.in.net# as the input network to the controller and the
output \verb#var.out.net# as the output network of the controller. 

\begin{verbatim}
var = somecontroller (in.settings)
controller(var.in.net,var.out.net)
\end{verbatim}

The output network is basically the new network setup, which means the
instantiated \verb#somecontroller# \verb#var# can add componenets and
connections.

\section{Documentation \& Comments}
It is possible to add documentation strings to components and their
inputs and outputs. The documentation string starts and ends with
\verb#'''#, and the documentation will be sent to The Copernicus
system.

\begin{verbatim}
atom python add   '''Add two floating point numbers: q=a+b'''
  in  ( float a
      , float b )
  out ( float o   '''a+b'''
      )
  options ( fuction : builtin.float.add
          , import  : builtin.float )
\end{verbatim}

The language has both line and comments block. A line comment start
with the symbol \# and anything after it will be skipped by the
parser. Comments block starts with /\# and ends with \#/. Anything
within the comment block will be skipped by the parser.

\begin{verbatim}
# This is a comment line

/#
This is a comment block
.
.
.
This is still in the comment block
/#
\end{verbatim}


\section{Future Work}

\subsection{Modules}
%A real functional import system is needed for the language to be
%realistically added to Copernicus. The developers will need to define
%how Copernicus should handle imports, since they have not decided
%what would be the most practicle and useful way to do this.

The current import system in Copernicus is to leave the import
mechanism to the server side. This may be discussed if it is optimal,
and the developers of Copernicus had some idées of how it could may be
changed or improved. Even though changing these mechanics does not
change the semantics for importing, it may affect the user of the DSL.

Importing standard libraries needs to be defined before the
client/server issues in Copernicus can be addressed. This may be an
easy problem but it is still vital for continuing the
implementation. Once this is done the developers needs to decide where
the code should be parsed and typechecked. The client can do all this
work and send it to the server but as the rest of the client is very
light and does not do much work, other than communicate with the
server, a prefered solution is to have the server do all the work. The
client can still add imports and build one code file for the server or
send all the used code files. As there are a difference between
importing from the standard library and importing user written project
specific files, it needs to be decided how the client and server will
work together to build the project. If the server should do most of
the work it needs to have all the libraries.

Assuming the server would get all user written files, building
projects, and doing all the work, a question is what the client should
be able to do. Users may for example be able to typecheck their
projects before sending them to a server, as it would make development
easier. If so, the client needs to have access to all the labraries
the server does.

There were a suggestion for a repository containing all libraries,
where code can be added and fetched when not available locally. Users
could easily add their own implementations. Such a repository needs to
handle executables aswell, since servers needs be able to supply them
to the workers.

\subsection{Transpose}
There is a need for a transpose function for types. The transpose
function would transform a record of any number of arrays to an array
of records. It would also work the other way around. Consider the
following two types:

\begin{verbatim}
type a ( int[]   : a0
       , float[] : a1 )

type b ( int   : b0
       , float : b1 )
\end{verbatim}

The code \verb#transpose(a)# would be the type \verb#b[]#, and the
code \verb#transpose(b[])# would be the type \verb#a#. The transpose
would work much like a zip for types which can take any kind of record
or array of record.
