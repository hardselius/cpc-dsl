\chapter{Language}\label{chap:language}
This chapter describes the domain-specific language
Codspeech\footnote{COpernicus Domain-SPEcific langagE, CHalmers}, as
well as the rationle behind the design.

For a complete description of the grammar in Backus-Naur Form (BNF),
see \autoref{sec:bnf}. The implementation details will be described
further in \autoref{chap:implementation}.


\section{Design}
Designing the DSL was a process which continued through the entire
project. As there are no text based solution to similar problems, the
DSL had no real starting base. The initial inspiration came from the
reasearch on programming praradigms and graphical implementations
related to network based programming.

Inspiration from well known programming languages was included for the
DSL to be simple and intuitive to the common user. Both functional and
imperative languages were considered when developing the design.

The most important steps in this process was a continuing discussion
with the developers of Copernicus. It was important to have a DSL
which they were satisfied with, but also to get input on what design
choices to make. The developers perspective was important for the DSL
to reflect realistic scenarios and to get a better collective view of
the different solutions. At each meeting the developers was presented
with a draft of the latest version of the DSL.


\section{General style and features}

\section{Modules}

\section{Typing}
\subsection{Primitive Types}
\subsection{Compound Types}
\subsection{New Records}
\subsection{Type System}

\section{Variables}

\section{Networks}

\section{Atoms}

\section{Statements}

\section{Expressions}

\section{Controllers}
