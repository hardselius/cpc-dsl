The idea behind an abstract syntax tree (AST) is to represent the
abstract syntactic structure of the source code in tree form. Each
node in the tree represents some structure occuring in the source. The
AST provides a good structure for later compiler stages since it omits
details having to do with the source language, and only contains
information about the essential structure of the program.

The AST is implemented using node classes for important language
constructs. All these node classes extends an abstract base
class. Since Python is dynamically typed, the concept of interfaces
does not really exist. Interfaces, commonly referred to as
``protocols'', are implicit. Determining these interfaces is based on
implementation introspection. The abstract base class looks like this:

\lstset{
  caption = {An abstract base class for AST nodes},
  label   = code:abstractnode
}
\begin{lstlisting}
class Node(object):
    """ Abstract base class for AST nodes.
    """
    def children(self):
        """ A sequence of all children that are Nodes
        """
        pass

    def show(
        self,
        buf=sys.stdout,
        offset=0,
        attrnames=False,
        nodenames=False,
        showcoord=False,
        _my_node_name=None):
        lead = ' ' * offset
        if nodenames and _my_node_name is not None:
            buf.write(
                lead + self.__class__.__name__+ ' <' + _my_node_name + '>: ')
        else:
            buf.write(lead + self.__class__.__name__+ ': ')

        if self.attr_names:
            if attrnames:
                nvlist = [(n, getattr(self,n)) for n in self.attr_names]
                attrstr = ', '.join('%s=%s' % nv for nv in nvlist)
            else:
                vlist = [getattr(self, n) for n in self.attr_names]
                attrstr = ', '.join('%s' % v for v in vlist)
            buf.write(attrstr)

        if showcoord:
            buf.write(' (at %s)' % self.coord)
        buf.write('\n')

        for (child_name, child) in self.children():
            child.show(
                buf,
                offset=offset + 2,
                attrnames=attrnames,
                nodenames=nodenames,
                showcoord=showcoord,
                _my_node_name=child_name)
\end{lstlisting}

\noindent This base class also contains a pretty printing function,
\texttt{show()}, that prints the entire tree below a the node from
which it was invoked from.

An AST node can be specified in the following way:

\lstset{
  caption = {Example of an AST node},
  label = code:samplenode
}
\begin{lstlisting}
class Header(Node):
    def __init__(self, ident, doc, inputs, outputs, coord=None):
        self.ident = ident
        self.doc = doc
        self.inputs = inputs
        self.outputs = outputs
        self.coord = coord

    def children(self):
        nodelist = []
        if self.ident is not None:
            nodelist.append(("ident", self.ident))
        if self.doc is not None:
            nodelist.append(("doc", self.doc))
        for i, child in enumerate(self.inputs or []):
            nodelist.append(("inputs[%d]" % i, child))
        for i, child in enumerate(self.outputs or []):
            nodelist.append(("outputs[%d]" % i, child))
        return tuple(nodelist)

    attr_names = ()
\end{lstlisting}

Python also does not support multiple dispatch at the language
definition or syntactic level, nor does it support method
overloading. However, the visitor pattern can be implemented using
method introspection. Another base class for visiting nodes is
defined:

\lstset{
  caption = {The NodeVisitor class},
  label   = code:nodevisitor
}
\begin{lstlisting}
class NodeVisitor(object):
    def visit(self, node):
        """ Visit a node.
        """
        method = 'visit_' + node.__class__.__name__
        visitor = getattr(self, method, self.generic_visit)
        return visitor(node)

    def generic_visit(self, node):
        """ Called if no explicit visitor function exists for a
            node. Implements preorder visiting of the node.
        """
        for c_name, c in node.children():
            self.visit(c)
\end{lstlisting}

\lstset{
  caption = {An example use of the NodeVisitor},
  label   = code:visitorexample
}
\begin{lstlisting}
class ConstantVisitor(NodeVisitor):
    def __init__(self):
        self.values = []

    def visit_Constant(self, node):
        self.values.append(node.value)

...

cv = ConstantVisitor()
cv.visit(node)
\end{lstlisting}
