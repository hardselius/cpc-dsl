Rheos has a quite interesting type system, which makes type-checking a
non-trivial task. The type-checker class will take the resulting AST
from the parser and use the visitor pattern to traverse the tree,
taking appropriate actions at every node while building an
environment.

\paragraph{Type-checker}
When type-checking Rheos, most of the steps are straight-forward, but
there are some cases where it becomes very complicated. Primitive
types and literals already contain their type information from the
parser stage. Type-checking of new types is a question of checking
their elements and adding the definition to the environment, in order
to make them recognizable to the rest of the program.

\emph{Parametrized} components typically can not be gradually
instantiated, but this is only partially true for Rheos, since
components can be instantiated without any inputs and have them
connected afterwards. On the other hand, components that require
meta-arguments must have all of them supplied at instantiation.

Resolving meta types can only be done when a component requiring
meta-type arguments is instantiated. A copy of the referred component
is placed in the local context and given a new name. The meta
arguments are type-checked to make sure they are of the same meta type
as the required arguments. If they are of the wrong type, or the
number of arguments given does not match the number of arguments
required, the type-checker raises an exception. If all these checks
are passed, the type-checker continues with retrieving the type of the
argument, and makes a variable substitution on the types of the
instantiated component. When this is done, type-checking of the
substituted type expression is resumed as if it were in the middle of
checking an ordinary type expression.


\paragraph{Environment}
The current environment is implemented to mimic the structure of the
actual Rheos code. Components, new types and instantiated components
are all stored in a record of \emph{\{key : value\}}-pairs. The keys
are the names of the entry, while the value is a representation of
their types. Types are generalized to a couple of different objects;
\texttt{Component, Newtype, Generictype} and \texttt{Type}. These
objects are instantiated and added to the environment by the
type-checker in a way that makes it possible to reference their
elements using ordinary object operations, kind of like how it is done
in the Rheos language. For example, to fetch the type \texttt{t} of
input-parameter \texttt{a} from component \texttt{comp}, stored in the
environment, one would write something like:

\lstset{
  caption = {Look-up of types in the environment},
  label   = code:envlookup
}
\begin{lstlisting}
# fetch the type of comp.in.a
t0 = env['comp'].inp.a

# fetch the type of an element from a new type
# type setting (
#   int[] a ,
#   float b
#   )
t1 = env['setting'].a
\end{lstlisting}
