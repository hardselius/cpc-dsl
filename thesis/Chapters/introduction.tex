\chapter{Introduction}
With the ever increasing need for storage and computational power,
governments, research institutes and industry are rushing to adopt
Cloud Computing, moving away from a model where computational projects
are executed on local computers.

The communities of researchers that need access to the computational
power required to carry out non-trivial simulations and analysis of
data are often distributed geographically, as are the computing
resources they rely on.

\section{Background}

%Cloud Computing and Grid Computing 360-Degree Compared:

%''Nevertheless,yes: the problems are mostly the same in Clouds and
%Grids. There is a common need to be able to manage large facilities;
%to define methods by which consumers discover, request, and use
%resources provided by the central facilities; and to implement the
%often highly parallel computations that execute on those resources.''

%''Provenance is still an unexplored area in CLoud environments, in
%which we need to deal with even more challenging issues such as
%tracking data production across different service providers (with
%different platform visibility and access policies) and across
%different software and hardware abstraction layers within on
%provider.''

\section{Problem statement}

\subsection{Delimitations}
\subsection{Purpose}

\section{Literature}


\section{Related work}

\section{Remaining chapters}

