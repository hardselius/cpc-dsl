\chapter{Introduction}
With the ever increasing need for storage and computational power,
governments, research institutes and industry are rushing to adopt
Cloud Computing, moving away from a model where computational projects
are executed on local computers.

The communities of researchers that need access to the computational
power required to carry out non-trivial simulations and analysis of
data are often distributed geographically, as are the computing
resources they rely on.

\section{Background}

%Cloud Computing and Grid Computing 360-Degree Compared:

%''Nevertheless,yes: the problems are mostly the same in Clouds and
%Grids. There is a common need to be able to manage large facilities;
%to define methods by which consumers discover, request, and use
%resources provided by the central facilities; and to implement the
%often highly parallel computations that execute on those resources.''

%''Provenance is still an unexplored area in CLoud environments, in
%which we need to deal with even more challenging issues such as
%tracking data production across different service providers (with
%different platform visibility and access policies) and across
%different software and hardware abstraction layers within on
%provider.''

\section{Problem statement}
The objective is to find and implement a solution for the need of a
new way of giving Copernicus information of the users projects. The
developers specifically stated that the wanted a domain-specific
language(DSL) for this solution, and that they later on want to add a
graphical solution using this DSL.

The DSL should allow users of Copernicus to define their computational
projects. The projects should be able to be defined as piping
computations in a data-flow network, which means that the DSL needs to
be able to describe data-flow networks in plain text.

The intended users are assumed to possess some knowledge of
programming, but are not necessarily adept programmers. The design of
the DSL should therefore be simple and intuitive. The DSL needs to be
easy to understand so it becomes an asset instead of an hindrance.

The DSL should be fully functional in Copernicus. The users needs to
be able to use all the features and properties available in
Copernicus.

Copernicus has plug-in libraries which needs to be usable in the
language. This implies a certain amount of flexibility since there are
not a static amount of plug-ins, as new ones can be added. The DSL
should be able to cope with any new plug-ins.

The implementation should have an output of a form so that it can
easilly be integrated in the Copernicus system. The implementation
also needs to be easy to install on any system, supercomputer or
other.

\subsection{Delimitations}
The most important part of the project is to have a working
implementation. There are features which can be added to the DSL for
describing even more advanced project with better syntax, e.g. simple
arithmetics.

XML GENERATION vs USING THE AST

This project has not looked at a graphical solution at all, only other
implementations for inspiration for the DSL. A graphical interface
would be good addition and such a solution can use most parts of this
project.

The language we chose to write the implementation in is Python. This
choice was based on formost an easy implementation and maintenance,
since Copernicus is written in Python. It would have been possible to
use effective tools and another language, but instead tools
specifically for python where used.


\subsection{Purpose}

\section{Literature}


\section{Related work}

\section{Remaining chapters}

