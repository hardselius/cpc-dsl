\chapter{Implementation}\label{chap:implementation}


%\section{Implementation}
%The implementation stage started when the first design prototype was
%done. This stage was developed parallel to the design, as both parts
%influenced eachother.
%
%The first step was to write a parser for building abstract syntax
%trees. The objective was to write a BNF grammar which represented our
%design. The grammar contains definitions of statements, expressions,
%types, etc.
%
%A type checker was the next step to be implemented. A type checker can
%provide the user with more expressive error reporting, and will also
%facilitate the learning experience.
%
%The last step was the XML generation. As this step is the one
%connecting the project to Copernicus, it required a more detailed
%understanding of how the system described and used computational
%projects.

\section{Tools}
This section will describe the tools that were used to implement the
Rheos language. It will serve not only as documentation for the
language, but also as a reference for someone who might want to create
their own language using the the same tools.

\subsection{Python}
The implementation language used for building Rheos was
python. However, python was not the first language considered. Other
languages considered were \emph{C, C++, C\#, F\#, Haskell} and
\emph{Java}. However, C\# and F\# was never really an option, since
both implementation, compilation and execution of code were tied to
Unix environments. The reason behind the consideration of the other
languages was that they are all supported by the \emph{BNF Converter}
(BNFC) \citep{bnfc:online}. BNFC is a compiler construction for
generating a compiler front-end from a \emph{Labeled BNF
  grammar}. Given this grammar, the tool produces
\begin{inparaenum}[(1)]
\item an abstract syntax implementation;
\item a case skeleton for the abstract syntax in the same language;
\item a lexer generator file;
\item a parser generator file;
\item a pretty-printer module;
\item a \LaTeX~file containing a specification of the language
\end{inparaenum} \citep{bnfc:online}.
While a compiler generator certainly would have made the
implementation a lot easier, the Copernicus system is written in
python, and python does not exist as a target for the BNF Converter.

Copernicus is designed to run on Unix machines with as few
dependencies as possible, which makes Java an unsuitable candidate,
since it cannot be assumed that every candidate node has a Java
runtime environment.

Doing the implementation in the C language could have been a possible
solution, since it integrates well with python. Most Unix system does
indeed ship with \emph{gcc} or the \emph{GNU Compiler
  Collection}. However, some older Unix distributions will not have
gcc pre-installed, and others like recent versions of \emph{Solaris}
and \emph{OpenSolaris} will have gcc under a different location.

Haskell was ruled out due to the simple fact that it is not as
mainstream as the other languages. While Haskell is a very powerful
language for writing compilers, maintenance of the code base might
prove difficult for inexperienced users.

Virtually every Unix system ships with a python interpreter, and it is
natural to write python extensions to a system already written in
Python. Python is easy to learn and the code is easy to extend and
maintain. In spite of python not being a classical meta-programming
language, it became the implementation language of choice.


\subsection{PLY (Python Lex-Yacc)}\label{sec:ply}
PLY is an implementation of \texttt{lex} and \texttt{yacc} parsing
tools, written purely in python, by \citet{ply:online}. It was
originally developed for an introductory class on compilers back in
2001. It provides most of the standard lex/yacc features including
support for empty productions, precedence rules, error recovery, and
support for ambiguous grammars. It uses LR-parsing, which is a
reasonable parsing scheme for larger grammars, but slightly restricts
the type of grammars that can be written \citep{aho:2007}. PLY is
straight-forward to use, and one its many advantages is
the \emph{very} extensive error checking, which certainly makes life
easier.

\paragraph{Python Lex}
The first step to implement the language is to write a tokenizer. This
is done with the Lex module of PLY. Language tokens are recognized
using regular expressions, and the steps are straightforward.

The names of all the token types are declared as a list of strings
named \texttt{tokens}.

\lstset{
  caption = {The token list},
  label   = code:tokenlist
}
\begin{lstlisting}
class RheosLexer(object):

...

    tokens = [
        # Literals: identifier, type, integer constant, float
        # constant, string constant
        'IDENT', 'ICONST', 'FCONST', 'SCONST', 'DOCSTRING',

        # Assignments: = :
        'EQUALS', 'COLON',

        # Connection: <-
        'CONNECTION',

        # Delimiters: ( ) { } [ ] , .
        'LPAREN', 'RPAREN',
        'LBRACE', 'RBRACE',
        'LBRACKET', 'RBRACKET',
        'COMMA', 'PERIOD',

        # Other:
        'CR', 'OPTIONAL', 'OPTIONS'
    ]
\end{lstlisting}

Tokens that require no special processing are declared using
module-level variables prefixed by \texttt{t_}, where the name
following \texttt{t_} has to exactly match some string in the tokens
list. Each such variable contains a regular expression string that
matches the respective token (Python raw strings are usually used
since they are the most convenient way to write regular expression
strings).

\lstset{
  caption = {Token variables},
  label   = code:tokenvar
}
\begin{lstlisting}
class RheosLexer(object):

...

    t_EQUALS     = r'='
    t_COLON      = r':'
    t_CONNECTION = r'<-'
    t_LPAREN     = r'\('
    t_RPAREN     = r'\)'
    t_LBRACKET   = r'\['
    t_RBRACKET   = r'\]'
    t_LBRACE     = r'\{'
    t_RBRACE     = r'\}'
    t_COMMA      = r','
    t_PERIOD     = r'\.'
    t_OPTIONAL   = r'\?'
\end{lstlisting}

When tokens do require special processing, a token rule can be
specified as a function. For example, this rule matches numbers and
converts the string into a Python integer.

\lstset{
  caption = {Token functions},
  label   = code:tokenfunc 
}
\begin{lstlisting}
    def t_ICONST(self, t):
        r'\d+'
        t.value = int(t.value)
        return t
\end{lstlisting}

In some cases, we may want to build tokens from more complex regular
expressions. For example:

\lstset{
  caption = {Complex regular expressions},
  label   = code:regex
}
\begin{lstlisting}
class RheosLexer(object):

...

    lowercase    = r'[a-z]'
    identchar    = r'[_A-Za-z0-9-]'
    ident        = r'(' + lowercase + r'(' + identchar + r')*)'

    def t_IDENT(self, t):
        # we want the doc-string to be the identifier above
        ...
\end{lstlisting}

\noindent This is not possible to specify using a normal doc-string. The
programmer would have to write the full RE, defeating the purpose of
re-usable code. However, there is a way around this by using the
\texttt{@TOKEN} decorator.

\lstset{
  caption = {Token decoratior},
  label   = code:token
}
\begin{lstlisting}
from ply.lex import TOKEN

class CodspeechLexer(object):

...


    lowercase    = r'[a-z]'
    identchar    = r'[_A-Za-z0-9-]'
    ident        = r'(' + lowercase + r'(' + identchar + r')*)'

    @TOKEN(ident)
    def t_IDENT(self, t):
        t.type = self.keyword_map.get(t.value,"IDENT")
        return t
\end{lstlisting}

The observant reader might notice something special going on in the
function \texttt{t_IDENT}. The processed string is checked against a
keyword map to decide whether the token type should actually be
\texttt{IDENT} or something else. The keyword map is defined as a
dictionary, and the values are appended to the token list.

\lstset{
  caption = {Keyword map},
  label   = {code:keywordmap}
}
\begin{lstlisting}
class RheosLexer(object):

...

    keyword_map = {
        # Import
        'import'          : 'IMPORT',

        # Type
        'type'            : 'TYPE',

        # Atom keywords
        'atom'            : 'ATOM',
        #'options'         : 'OPTIONS',
        'python'          : 'ATOMTYPE',
        'python-extended' : 'ATOMTYPE',
        'external'        : 'ATOMTYPE',

        # Network
        'network'         : 'NETWORK',
        'controller'      : 'CONTROLLER',
                
        # Header
        'in'              : 'IN',
        'out'             : 'OUT',
        'default'         : 'DEFAULT',
        
        # Types
        'file'            : 'FILE',
        'float'           : 'FLOAT',
        'int'             : 'INT',
        'string'          : 'STRING',
    }


    tokens = [
        ...
    ] + list(set(keyword_map.values()))
\end{lstlisting}

\noindent Since our keyword map contains multiple keys mapping to the
same value and the token list can not contain any duplicates, the list
of values is converted to a set before it is converted back into a
list.


\paragraph{Python Yacc}
The \texttt{yacc.py} module is used to parse the language
syntax. \emph{Syntax} is usually specified in terms of a
\emph{BNF-grammar} (\highlight{citation needed}). For example, some
simple grammar rules for parsing types could look like this:

\begin{figure}[h!]
  \begin{grammar}
    <type> ::= `float'
    \alt `int'
    \alt `string'
    \alt <type> <dim>

    <dim> ::= `[]'
    \alt <dim> `[]'
  \end{grammar}
  \caption{An example grammar for type identifiers}
  \label{grammar:typeex}
\end{figure}

\noindent The identifiers \emph{type} and \emph{dim} refer to grammar
rules comprised of a collection of \emph{terminals} and
\emph{non-terminals}. The symbols \texttt{float}, \texttt{int},
\texttt{string} and \texttt{[]} are known as the \emph{terminals}
and correspond to raw input tokens. The \emph{non-terminals}, such as
\emph{dim}, refer to other rules.

The \emph{semantic} behavior of a language is often specified using
syntax directed translation. Each symbol in a given grammar rule has a
set of attributes associated with them along with an action. The
action describes what to do whenever a particular grammar rule is
recognized.

Yacc uses a parsing technique called lookahead-LR (LALR) parsing,
which is based on the LR(0) sets of items, but has fewer states than
typical parsers based on the LR(1) items \citep{aho:2007}. It is a
bottom up scheme that tries to match a sequence of lexical objects
against the right-hand-side of various grammar rules. Whenever a
matching right-hand-side is found, the appropriate action code is
triggered and the grammar symbols are replaced by the grammar symbol
on the left-hand-side.

Implementing a parser in Python Yacc is fairly straight-forward. The
list of tokens from the lexer module is imported and a series of
functions describing the grammar productions are defined. From the
grammar in \myref{figure}{grammar:typeex} the corresponding Python
code becomes:

\lstset{
  caption = {Parser example},
  label   = code:parser
}
\begin{lstlisting}
    def p_type(self, p):
        """
        type : FILE
             | FLOAT
             | INT
             | STRING
             | IDENT
             | type dim
        """
        if len(p) == 2:
            p[0] = csast.Type(p[1])
        else:
            p[1].type += p[2]
            p[0] = p[1]


    def p_dim(self, p):
        """
        dim : LBRACKET RBRACKET
            | LBRACKET RBRACKET dim
        """
        if len(p) == 3:
            p[0] = '[]'
        else:
            p[0] = '[]' + p[3]
\end{lstlisting}

Each function has a doc string that contains the appropriate
context-free grammar specification. This idea was actually borrowed
from the SPARK toolkit \citep{spark:online}. A function takes an
argument, \emph{p}, that contains a sequence, starting at index 1, of
values matching the symbols in the corresponding rule. The value
\texttt{p[0]} is mapped to the left-hand-side rule, while the values
in \texttt{p[1..]} are mapped to the grammar symbols on the
right-hand-side. The statements in the function body implements the
semantic actions of the rule. In this case, we use the parser to to
build an abstract syntax tree. This is described in more detail in
\myref{section}{sec:ast}.


\paragraph{Alternative specification of Lexer and Parser}
As seen in the above examples, both the lexer and parser are defined
from instances of their own classes. The easiest way, however, is to
specify them directly in their own modules. The PLY documentation
explains this quite well, complete with examples \citep{ply:online}.



\section{Implementation details}
This section will describe the various implementation steps taken
during the construction of Rheos.


\subsection{Abstract Syntax Tree}\label{sec:ast}
The idea behind an abstract syntax tree (AST) is to represent the
abstract syntactic structure of the source code in tree form. Each
node in the tree represents some structure occuring in the source. The
AST provides a good structure for later compiler stages since it omits
details having to do with the source language, and only contains
information about the essential structure of the program.

The AST is implemented using node classes for important language
constructs. All these node classes extends an abstract base
class. Since Python is dynamically typed, the concept of interfaces
does not really exist. Interfaces, commonly referred to as
``protocols'', are implicit. Determining these interfaces is based on
implementation introspection. The abstract base class looks like
this \citep{pycparser:online}:

\lstset{
  caption = {An abstract base class for AST nodes},
  label   = code:abstractnode
}
\begin{lstlisting}
class Node(object):
    """ Abstract base class for AST nodes.
    """
    def children(self):
        """ A sequence of all children that are Nodes
        """
        pass

    def show(
        self,
        buf=sys.stdout,
        offset=0,
        attrnames=False,
        nodenames=False,
        showcoord=False,
        _my_node_name=None):
        lead = ' ' * offset
        if nodenames and _my_node_name is not None:
            buf.write(
                lead + self.__class__.__name__+ ' <' + _my_node_name + '>: ')
        else:
            buf.write(lead + self.__class__.__name__+ ': ')

        if self.attr_names:
            if attrnames:
                nvlist = [(n, getattr(self,n)) for n in self.attr_names]
                attrstr = ', '.join('%s=%s' % nv for nv in nvlist)
            else:
                vlist = [getattr(self, n) for n in self.attr_names]
                attrstr = ', '.join('%s' % v for v in vlist)
            buf.write(attrstr)

        if showcoord:
            buf.write(' (at %s)' % self.coord)
        buf.write('\n')

        for (child_name, child) in self.children():
            child.show(
                buf,
                offset=offset + 2,
                attrnames=attrnames,
                nodenames=nodenames,
                showcoord=showcoord,
                _my_node_name=child_name)
\end{lstlisting}

\noindent This base class also contains a pretty printing function,
\texttt{show()}, that prints the entire tree below a the node from
which it was invoked from.

An AST node can be specified in the following way:

\lstset{
  caption = {Example of an AST node},
  label = code:samplenode
}
\begin{lstlisting}
class Header(Node):
    def __init__(self, ident, doc, inputs, outputs, coord=None):
        self.ident = ident
        self.doc = doc
        self.inputs = inputs
        self.outputs = outputs
        self.coord = coord

    def children(self):
        nodelist = []
        if self.ident is not None:
            nodelist.append(("ident", self.ident))
        if self.doc is not None:
            nodelist.append(("doc", self.doc))
        for i, child in enumerate(self.inputs or []):
            nodelist.append(("inputs[%d]" % i, child))
        for i, child in enumerate(self.outputs or []):
            nodelist.append(("outputs[%d]" % i, child))
        return tuple(nodelist)

    attr_names = ()
\end{lstlisting}

Python also does not support multiple dispatch at the language
definition or syntactic level, nor does it support method
overloading. However, the visitor pattern can be implemented using
method introspection. Another base class for visiting nodes is
defined:

\lstset{
  caption = {The NodeVisitor class},
  label   = code:nodevisitor
}
\begin{lstlisting}
class NodeVisitor(object):
    def visit(self, node):
        """ Visit a node.
        """
        method = 'visit_' + node.__class__.__name__
        visitor = getattr(self, method, self.generic_visit)
        return visitor(node)

    def generic_visit(self, node):
        """ Called if no explicit visitor function exists for a
            node. Implements preorder visiting of the node.
        """
        for c_name, c in node.children():
            self.visit(c)
\end{lstlisting}

\lstset{
  caption = {An example use of the NodeVisitor},
  label   = code:visitorexample
}
\begin{lstlisting}
class ConstantVisitor(NodeVisitor):
    def __init__(self):
        self.values = []

    def visit_Constant(self, node):
        self.values.append(node.value)

...

cv = ConstantVisitor()
cv.visit(node)
\end{lstlisting}



\subsection{Type-checker}\label{sec:typechecker}
Rheos has a quite interesting type system, which makes type-checking a
non-trivial task. The type-checker class will take the resulting AST
from the parser and use the visitor pattern to traverse the tree,
taking appropriate actions at every node while building an
environment.

\paragraph{Type-checker}
When type-checking Rheos, most of the steps are straight-forward, but
there are some cases where it becomes very complicated. Primitive
types and literals already contain their type information from the
parser stage. Type-checking of new types is a question of checking
their elements and adding the definition to the environment, in order
to make them recognizable to the rest of the program.

\emph{Parametrized} components typically can not be gradually
instantiated, but this is only partially true for Rheos, since
components can be instantiated without any inputs and have them
connected afterwards. On the other hand, components that require
meta-arguments must have all of them supplied at instantiation.

Resolving meta types can only be done when a component requiring
meta-type arguments is instantiated. A copy of the referred component
is placed in the local context and given a new name. The meta
arguments are type-checked to make sure they are of the same meta type
as the required arguments. If they are of the wrong type, or the
number of arguments given does not match the number of arguments
required, the type-checker raises an exception. If all these checks
are passed, the type-checker continues with retrieving the type of the
argument, and makes a variable substitution on the types of the
instantiated component. When this is done, type-checking of the
substituted type expression is resumed as if it were in the middle of
checking an ordinary type expression.


\paragraph{Environment}
The current environment is implemented to mimic the structure of the
actual Rheos code. Components, new types and instantiated components
are all stored in a record of \emph{\{key : value\}}-pairs. The keys
are the names of the entry, while the value is a representation of
their types. Types are generalized to a couple of different objects;
\texttt{Component, Newtype, Generictype} and \texttt{Type}. These
objects are instantiated and added to the environment by the
type-checker in a way that makes it possible to reference their
elements using ordinary object operations, kind of like how it is done
in the Rheos language. For example, to fetch the type \texttt{t} of
input-parameter \texttt{a} from component \texttt{comp}, stored in the
environment, one would write something like:

\lstset{
  caption = {Look-up of types in the environment},
  label   = code:envlookup
}
\begin{lstlisting}
# fetch the type of comp.in.a
t0 = env['comp'].inp.a

# fetch the type of an element from a new type
# type setting (
#   int[] a ,
#   float b
#   )
t1 = env['setting'].a
\end{lstlisting}



\subsection{XML generation}\label{sec:xml}
There is an implemented XML generator for an earlier version of
Rheos. Due to signifacant changes of the language description, other
aspects were prioritized and XML generation was left for the
developers of Copernicus to update. The new language description that
emerged was in fact so different from the original, that Copernicus
needed updates to incorporate those changes.

The XML generator is implemented using the same visitor pattern as the
type-checker. Visiting the different nodes in the abstract syntax tree
produces corresponding XML code, and traversing the while tree will
yield an entire definition, complete with indentations.


\subsection{Emacs mode}\label{sec:emacs}
The Emacs mode provides nothing more than syntax highlighting. The
mode was created mainly to provide a more appealing look to the
example code developed during design and testing of the Rheos. The
syntax highlighting adds some understanding of what the code actually
represents, which made it easier to add and change specific parts of
the DSL.


\section{Future work}
\subsection{Lexer and Parser}
As described in \autoref{sec:transpose}, there is a need for a
\texttt{transpose} primitive. This would have to be specified as a
special keyword in the lexer and also be addressed separately in the
parser.


\subsection{Type-checker}
What remains to be done and future work implementation-wise has a lot
to do with the type-checker. Since it was decided to add polymorphic
stage in the project, the type-checker had to be completely
parametrized typing to allow for generic components at a very late
re-written. This change proved to be very time-consuming.

The environment was re-written at the same time to make it more
powerful and intuitive to work with. Before the current
implementation, the environment consisted of a lot of different
records and lists, and did not perform well on look-ups.
