\chapter{Implementation}\label{chap:implementation}


%\section{Implementation}
%The implementation stage started when the first design prototype was
%done. This stage was developed parallel to the design, as both parts
%influenced eachother.
%
%The first step was to write a parser for building abstract syntax
%trees. The objective was to write a BNF grammar which represented our
%design. The grammar contains definitions of statements, expressions,
%types, etc.
%
%A type checker was the next step to be implemented. A type checker can
%provide the user with more expressive error reporting, and will also
%facilitate the learning experience.
%
%The last step was the XML generation. As this step is the one
%connecting the project to Copernicus, it required a more detailed
%understanding of how the system described and used computational
%projects.

\section{Tools}

\subsection{Python}
The implementation language used for building Codspeech was
python. However, python was not the first language considered. Other
languages considered were \emph{C, C++, C\#, F\#, Haskell} and
\emph{Java}. However, C\# and F\# was never really an option, since
both implemenation, compilation and execution of code were tied to
Unix environments. The reason behind the consideration of the other
languages was that they are all supported by the \emph{BNF Converter}
(BNFC) \citep{bnfc:online}. BNFC is a compiler construction for
generating a compiler front-end from a \emph{Labeled BNF
  grammar}. Given this grammar, the tool produces
\begin{inparaenum}[(1)]
\item an abstract syntax implementation;
\item a case skeleton for the abstract syntax in the same language;
\item a lexer generator file;
\item a parser generator file;
\item a pretty-printer module;
\item a \LaTeX file containing a specification of the language
\end{inparaenum} \citep{bnfc:online}.
While a compiler generator certainly would have made the
implementation alot easier, the Copernicus system is written in
python, and python does not exist as a target for the BNF Converter.

Copernicus is designed to run on Unix machines with as few
dependencies as possible, which makes Java an unsuitable candidate --
there are possible difficulties of installing a Java runtime
environment on supercomputers.

Doing the implementation in the C language could have been a possible
solution, since it integrates weel with python. Most Unix system does
indeed ship with \emph{gcc} or the \emph{GNU Compiler
  Collection}. However, some older Unix distros will not have gcc
preinstalled, and others like recent versions of \emph{Solaris} and
\emph{OpenSolaris} will have gcc under a different location.

Haskell was ruled out due to the simple fact that it is not as
mainstream as the other languages. While Haskell is a very powerful
language for writing compilers, maintenance of the code base might
prove difficult for unexperienced users.

Virtually every Unix system ships with a python interpreter, and it is
natural to write python extensions to a system already written in
Python. Python is easy to learn and the code is easy to extend and
maintain. In spite of python not being a classical meta-programming
language, it became the implementation language of choice.


\subsection{PLY (Python Lex-Yacc)}\label{sec:ply}
PLY is an implementation of \texttt{lex} and \texttt{yacc} parsing
tools, written purely in python, by \citet{ply:online}. It was
originally developed for an introductory class on compilers back in
2001. It provides most of the standard lex/yacc features including
support for empty productions, precedence rules, error recovery, and
support for ambigous grammars. It uses LR-parsing, which is a
reasonabe parsing scheme for larger grammars, but slightly restricts
the type of grammars that can be written \citep{aho:2007}. PLY is
straight-forward to use, and one its many advantages is
the \emph{very} extensive error checking, which certainly makes life
easier.

\paragraph{Python Lex}
The first step to implement the language is to write a tokenizer. This
is done with the Lex module of PLY. Language tokens are recognized
using regular expressions, and the steps are straighforward.

The names of all the token types are declard as a list of strings
named \texttt{tokens}.

\lstset{
  caption = {The token list},
  label   = code:tokenlist
}
\begin{lstlisting}
class CodspeechLexer(object):

...

    tokens = [
        # Literals: identifier, type, integer constant, float
        # constant, string constant
        'IDENT', 'ICONST', 'FCONST', 'SCONST', 'DOCSTRING',

        # Assignments: = :
        'EQUALS', 'COLON',

        # Connection: <-
        'CONNECTION',

        # Delimeters: ( ) { } [ ] , .
        'LPAREN', 'RPAREN',
        'LBRACE', 'RBRACE',
        'LBRACKET', 'RBRACKET',
        'COMMA', 'PERIOD',

        # Other:
        'CR', 'OPTIONAL', 'OPTIONS'
    ]
\end{lstlisting}

Tokens that require no special processing are declared using
module-level variables prefixed by \texttt{t_}, where the name
following \texttt{t_} has to exactly match some string in the tokens
list. Each such variable contains a regular expression string that
matches the respective token (Python raw strings are usually used
since they are the most convenient way to write regular expression
strings).

\lstset{
  caption = {Token variables},
  label   = code:tokenvar
}
\begin{lstlisting}
class CodspeechLexer(object):

...

    t_EQUALS     = r'='
    t_COLON      = r':'
    t_CONNECTION = r'<-'
    t_LPAREN     = r'\('
    t_RPAREN     = r'\)'
    t_LBRACKET   = r'\['
    t_RBRACKET   = r'\]'
    t_LBRACE     = r'\{'
    t_RBRACE     = r'\}'
    t_COMMA      = r','
    t_PERIOD     = r'\.'
    t_OPTIONAL   = r'\?'
\end{lstlisting}

When tokens do require special processing, a token rule can be
specified as a function. For example, this rule matches numbers and
converts the string into a Python integer.

\lstset{
  caption = {Token functions},
  label   = code:tokenfunc 
}
\begin{lstlisting}
    def t_ICONST(self, t):
        r'\d+'
        t.value = int(t.value)
        return t
\end{lstlisting}

In some cases, we may want to build tokens from more complex regular
expressions. For example:

\lstset{
  caption = {Complex regular expressions},
  label   = code:regex
}
\begin{lstlisting}
class CodspeechLexer(object):

...

    lowercase    = r'[a-z]'
    identchar    = r'[_A-Za-z0-9-]'
    ident        = r'(' + lowercase + r'(' + identchar + r')*)'

    def t_IDENT(self, t):
        # we want the docstring to be the identifier above
        ...
\end{lstlisting}

\noindent This is not possible to specify using a normal docstring. The
programmer would have to write the full RE, defeating the purpose of
re-usable code. However, there is a way around this by using the
\texttt{@TOKEN} decorator.

\lstset{
  caption = {Token decoratior},
  label   = code:token
}
\begin{lstlisting}
from ply.lex import TOKEN

class CodspeechLexer(object):

...


    lowercase    = r'[a-z]'
    identchar    = r'[_A-Za-z0-9-]'
    ident        = r'(' + lowercase + r'(' + identchar + r')*)'

    @TOKEN(ident)
    def t_IDENT(self, t):
        t.type = self.keyword_map.get(t.value,"IDENT")
        return t
\end{lstlisting}

The observant reader might notice something special goind on in the
function \texttt{t_IDENT}. The processed string is checked against a
keyword map to decide wheter the token type should actually be
\texttt{IDENT} or something else. The keyword map is defined as a
dictionary, and the values are appended to the token list.

\lstset{
  caption = {Keyword map},
  label   = {code:keywordmap}
}
\begin{lstlisting}
class CodspeechLexer(object):

...

    keyword_map = {
        # Import
        'import'          : 'IMPORT',

        # Type
        'type'            : 'TYPE',

        # Atom keywords
        'atom'            : 'ATOM',
        #'options'         : 'OPTIONS',
        'python'          : 'ATOMTYPE',
        'python-extended' : 'ATOMTYPE',
        'external'        : 'ATOMTYPE',

        # Network
        'network'         : 'NETWORK',
        'controller'      : 'CONTROLLER',
                
        # Header
        'in'              : 'IN',
        'out'             : 'OUT',
        'default'         : 'DEFAULT',
        
        # Types
        'file'            : 'FILE',
        'float'           : 'FLOAT',
        'int'             : 'INT',
        'string'          : 'STRING',
    }


    tokens = [
        ...
    ] + list(set(keyword_map.values()))
\end{lstlisting}

\noindent Since our keyword map contains multiple keys mapping to the
same value and the token list can not contain any duplicates, the list
of values is converted to a set before it is converted back into a
list.


\paragraph{Python Yacc}
The \texttt{yacc.py} module is used to parse the language
syntax. \emph{Syntax} is usually specified in terms of a
\emph{BNF-grammar} (\highlight{citation needed}). For example, some
simple grammar rules for parsing types could look like this:

\begin{figure}[h!]
  \begin{grammar}
    <type> ::= `float'
    \alt `int'
    \alt `string'
    \alt <type> <dim>

    <dim> ::= `[]'
    \alt <dim> `[]'
  \end{grammar}
  \caption{An example grammar for type identifiers}
  \label{grammar:typeex}
\end{figure}

\noindent The identifiers \emph{type} and \emph{dim} refer to grammar
rules comprised of a collection of \emph{terminals} and
\emph{non-terminals}. The symbols \texttt{float}, \texttt{int},
\texttt{string} and \texttt{[]} are known as the \emph{terminals}
and correspond to raw input tokens. The \emph{non-terminals}, such as
\emph{dim}, refer to other rules.

The \emph{semantic} behaviour of a language is often specified using
syntax directed translation. Each symbol in a given grammar rule has a
set of attributes associated with them along with an action. The
action describes what to do whenever a particular grammar rule is
recognized.

Yacc uses a parsing technique called LR-parsing, also known as
\emph{shift-reduce} parsing. It is a bottom up scheme that tries to
match a sequence of lexical objects against the right-hand-side of
various grammar rules. Whenever a matching right-hand-side is found,
the appropriate action code is triggered and the grammar symbols are
replaced by the grammar symbol on the left-hand-side.

Implementing a parser in Python Yacc is fairly straight forward. The
list of tokens from the lexer module is imported and a series of
functions describing the grammar productions are defined. From the
gramar in \autoref{grammar:typeex} the corresponding Python code
becomes:

\lstset{
  caption = {Parser example},
  label   = code:parser
}
\begin{lstlisting}
    def p_type(self, p):
        """
        type : FILE
             | FLOAT
             | INT
             | STRING
             | IDENT
             | type dim
        """
        if len(p) == 2:
            p[0] = csast.Type(p[1])
        else:
            p[1].type += p[2]
            p[0] = p[1]


    def p_dim(self, p):
        """
        dim : LBRACKET RBRACKET
            | LBRACKET RBRACKET dim
        """
        if len(p) == 3:
            p[0] = '[]'
        else:
            p[0] = '[]' + p[3]
\end{lstlisting}

Each function has a doc string that contains the appropriate
context-free grammar specification. This idea was actually borrowed
from the SPARK toolkit \citep{spark:online}. A function takes an
argument, \emph{p}, that contains a sequence, starting at index 1, of
values matching the symbols in the corresponding rule. The value
\texttt{p[0]} is mapped to the left-hand-side rule, while the values
in \texttt{p[1..]} are mapped to the grammar symbold on the
right-hand-side. The statements in the function body implements the
semantic actions of the rule. In this case, we use the parser to to
build an abstract syntax tree. This is described in more detail in
\autoref{sec:ast}.


\paragraph{Alternative specification of Lexer and Parser}
As seen in the above examples, both the lexer and parser are defined
from instances of their own classes. The easiest way, however, is to
specify them directly in their own modules. The PLY documentation
explains this quite well, complete with examples \citep{ply:online}.



\section{Implementation details}
This section will describe the various implementation steps taken
during the construction of Codspeech.

\subsection{Abstract Syntax Tree}\label{sec:ast}
The idea behind an abstract syntax tree (AST) is to represent the
abstract syntactic structure of the source code in tree form. Each
node in the tree represents some structure occurring in the source. The
AST provides a good structure for later compiler stages since it omits
details having to do with the source language, and only contains
information about the essential structure of the program.

The AST is implemented using node classes for important language
constructs. All these node classes extends an abstract base
class. Since Python is dynamically typed, the concept of interfaces
does not really exist. Interfaces, commonly referred to as
``protocols'', are implicit. Determining these interfaces is based on
implementation introspection. The implementation of the abstract base
is given in \myref{listing}{code:abstractnode}
below \citep{pycparser:online}.

\lstset{
  caption = {An abstract base class for AST nodes},
  label   = code:abstractnode
}
\begin{lstlisting}
class Node(object):
    """ Abstract base class for AST nodes.
    """
    def children(self):
        """ A sequence of all children that are Nodes
        """
        pass

    def show(
        self,
        buf=sys.stdout,
        offset=0,
        attrnames=False,
        nodenames=False,
        showcoord=False,
        _my_node_name=None):
        lead = ' ' * offset
        if nodenames and _my_node_name is not None:
            buf.write(
                lead + self.__class__.__name__+ ' <' + _my_node_name + '>: ')
        else:
            buf.write(lead + self.__class__.__name__+ ': ')

        if self.attr_names:
            if attrnames:
                nvlist = [(n, getattr(self,n)) for n in self.attr_names]
                attrstr = ', '.join('%s=%s' % nv for nv in nvlist)
            else:
                vlist = [getattr(self, n) for n in self.attr_names]
                attrstr = ', '.join('%s' % v for v in vlist)
            buf.write(attrstr)

        if showcoord:
            buf.write(' (at %s)' % self.coord)
        buf.write('\n')

        for (child_name, child) in self.children():
            child.show(
                buf,
                offset=offset + 2,
                attrnames=attrnames,
                nodenames=nodenames,
                showcoord=showcoord,
                _my_node_name=child_name)
\end{lstlisting}

\noindent This base class also contains a pretty printing function,
\texttt{show()}, that prints the entire tree below a the node from
which it was invoked from.

An AST node can be specified in the following way:

\lstset{
  caption = {Example of an AST node},
  label = code:samplenode
}
\begin{lstlisting}
class Header(Node):
    def __init__(self, ident, doc, inputs, outputs, coord=None):
        self.ident = ident
        self.doc = doc
        self.inputs = inputs
        self.outputs = outputs
        self.coord = coord

    def children(self):
        nodelist = []
        if self.ident is not None:
            nodelist.append(("ident", self.ident))
        if self.doc is not None:
            nodelist.append(("doc", self.doc))
        for i, child in enumerate(self.inputs or []):
            nodelist.append(("inputs[%d]" % i, child))
        for i, child in enumerate(self.outputs or []):
            nodelist.append(("outputs[%d]" % i, child))
        return tuple(nodelist)

    attr_names = ()
\end{lstlisting}

Python also does not support multiple dispatch at the language
definition or syntactic level, nor does it support method
overloading. However, the visitor pattern can be implemented using
method introspection. Another base class for visiting nodes is
defined:

\lstset{
  caption = {The NodeVisitor class},
  label   = code:nodevisitor
}
\begin{lstlisting}
class NodeVisitor(object):
    def visit(self, node):
        """ Visit a node.
        """
        method = 'visit_' + node.__class__.__name__
        visitor = getattr(self, method, self.generic_visit)
        return visitor(node)

    def generic_visit(self, node):
        """ Called if no explicit visitor function exists for a
            node. Implements preorder visiting of the node.
        """
        for c_name, c in node.children():
            self.visit(c)
\end{lstlisting}

\lstset{
  caption = {An example use of the NodeVisitor},
  label   = code:visitorexample
}
\begin{lstlisting}
class ConstantVisitor(NodeVisitor):
    def __init__(self):
        self.values = []

    def visit_Constant(self, node):
        self.values.append(node.value)

...

cv = ConstantVisitor()
cv.visit(node)
\end{lstlisting}



\subsection{Codspeech BNF}\label{sec:bnf}
\newcommand{\pn}[1]{\langle \textnormal{#1} \rangle}
\newcommand{\pp}{\models}
\newcommand{\oo}{\; \mid \;}
\newcommand{\sk}{\dots }
\newcommand{\ww}{\;}
\newcommand{\nn}{\perp}
\newcommand{\sm}[1]{\textnormal{#1}}
\newcommand{\sd}[1]{\textnormal{\it #1}}


Grammar production
\begin{eqnarray*}
  \pn{entrypoint} &\pp&
    \pn{opt{\textunderscore}cr} \ww \pn{program} \oo
    \pn{empty} \\
  \pn{program} &\pp&
    \pn{top{\textunderscore}def{\textunderscore}list} \ww
    \pn{opt{\textunderscore}cr} \\
  \pn{top{\textunderscore}def{\textunderscore}list} &\pp&
    \pn{top{\textunderscore}def} \oo
    \pn{top{\textunderscore}def{\textunderscore}list} \ww
    \pn{cr} \ww
    \pn{top{\textunderscore}def} \\
  \pn{top{\textunderscore}def} &\pp&
    \pn{import{\textunderscore}stmt} \oo
    \pn{newtype{\textunderscore}decl} \oo
    \pn{atom{\textunderscore}decl} \oo
    \pn{network{\textunderscore}decl} \\
\end{eqnarray*}

Import Statment
\begin{eqnarray*}
  \pn{import{\textunderscore}stmt} &\pp&
    \sm{import} \ww
    \pn{package{\textunderscore}path} \\
  \pn{package{\textunderscore}path} &\pp& 
    \pn{package{\textunderscore}identifier} \oo
    \pn{package{\textunderscore}path} \ww
    \sm{.} \ww
    \pn{package{\textunderscore}identifier} \\
  \pn{package{\textunderscore}identifier} &\pp& \pn{ident} \\
\end{eqnarray*}

Newtype
\begin{eqnarray*}
  \pn{newtype{\textunderscore}decl} &\pp&
    \sm{type} \ww
    \pn{type} \ww
    \pn{docstring} \ww
    \pn{cr} \ww
    \sm{(} \ww
    \pn{type{\textunderscore}decl{\textunderscore}list} \ww
    \sm{)} \\
  \pn{type{\textunderscore}decl{\textunderscore}list} &\pp&
    \pn{type{\textunderscore}decl} \oo
    \pn{type{\textunderscore}decl{\textunderscore}list} \ww
    \pn{comma{\textunderscore}sep} \ww
    \pn{type{\textunderscore}decl} \\
  \pn{type{\textunderscore}decl} &\pp&
    \pn{type} \ww
    \sm{:} \ww
    \pn{ident} \\
\end{eqnarray*}

Atom Declaration
\begin{eqnarray*}
  \pn{atom{\textunderscore}decl} &\pp&
    \sm{atom} \ww
    \pn{atomtype} \ww
    \pn{header} \ww
    \pn{optionblock} \\
  \pn{atomtype} &\pp&
    \sm{python} \oo
    \sm{python-extended} \oo
    \sm{external} \oo \\
  \pn{optionblock} &\pp&
    \sm{option} \ww
    \pn{opt{\textunderscore}cr} \ww
    \sm{(} \ww
    \pn{atom{\textunderscore}optio{\textunderscore}list} \ww
    \sm{)} \\
  \pn{atom{\textunderscore}option{\textunderscore}list} &\pp&
    \pn{atom{\textunderscore}option} \oo
    \pn{atom{\textunderscore}option{\textunderscore}list} \ww
    \pn{comma{\textunderscore}sep} \ww
    \pn{atom{\textunderscore}option} \\
  \pn{atom{\textunderscore}option} &\pp&
    \pn{option{\textunderscore}ident} \ww
    \sm{:} \ww
    \pn{option{\textunderscore}ident} \ww\\
  \pn{option{\textunderscore}ident} &\pp&
    \pn{ident} \oo
    \pn{option{\textunderscore}ident} \ww
    \sm{.} \ww
    \pn{ident} \\
\end{eqnarray*}

Network Declaration
\begin{eqnarray*}
  \pn{network{\textunderscore}decl} &\pp&
    \sm{network} \ww
    \pn{header} \ww
    \pn{networkblock} \\
  \pn{networkblock} &\pp&
    \pn{stmt{\textunderscore}block} \\
\end{eqnarray*}

Atom and Network Header
\begin{eqnarray*}
  \pn{header} &\pp&
    \pn{ident} \ww
    \pn{docstring} \ww
    \pn{cr} \ww
    \pn{inputs} \ww
    \pn{outputs} \\
  \pn{inputs} &\pp&
    \sm{in} \ww
    \sm{(} \ww
    \pn{input{\textunderscore}list} \ww
    \sm{)} \ww
    \pn{cr} \oo
    \sm{in} \ww
    \pn{no{\textunderscore}params} \ww
    \pn{cr} \\
  \pn{outputs} &\pp&
    \sm{out} \ww
    \sm{(} \ww
    \pn{output{\textunderscore}list} \ww
    \sm{)} \ww
    \pn{cr} \oo
    \sm{out} \ww
    \pn{no{\textunderscore}params} \ww
    \pn{cr} \\
  \pn{input{\textunderscore}list} &\pp&
    \pn{input} \oo
    \pn{input{\textunderscore}list} \ww
    \pn{comma{\textunderscore}sep} \ww
    \pn{input} \\
  \pn{output{\textunderscore}list} &\pp&
    \pn{output} \oo
    \pn{output{\textunderscore}list} \ww
    \pn{comma{\textunderscore}sep} \ww
    \pn{output} \\
  \pn{no{\textunderscore}params} &\pp&
    \sm{(} \ww
    \pn{opt{\textunderscore}cr} \ww
    \sm{)} \\
  \pn{input} &\pp&
    \pn{type} \ww
    \pn{ident} \ww
    \pn{docstring} \oo
    \sm{?} \ww
    \pn{type} \ww
    \pn{ident} \ww
    \pn{docstring} \oo \\
    & & \pn{type} \ww
    \pn{ident} \ww
    \sm{default} \ww
    \pn{constant} \ww
    \pn{docstring} \\
  \pn{output} &\pp&
    \pn{type} \ww
    \pn{ident} \ww
    \pn{docstring} \oo \\
    & & \pn{type} \ww
    \pn{ident} \ww
    \sm{default} \ww
    \pn{constant} \ww
    \pn{docstring} \\
\end{eqnarray*}

Statements
\begin{eqnarray*}
  \pn{stmt{\textunderscore}block} &\pp&
    \pn{opt{\textunderscore}cr} \ww
    \pn{lbrace} \ww
    \pn{stmt{\textunderscore}list} \ww
    \pn{rbrace} \oo \\
    & & \pn{opt{\textunderscore}cr} \ww
    \pn{stmt{\textunderscore}block{\textunderscore}empty} \\
  \pn{stmt{\textunderscore}block{\textunderscore}empty} &\pp&
    \sm{(} \ww
    \pn{opt{\textunderscore}cr} \ww
    \sm{)} \\
  \pn{stmt{\textunderscore}list} &\pp&
    \pn{stmt} \oo
    \pn{stmt{\textunderscore}list} \ww
    \pn{cr} \ww
    \pn{stmt} \\
  \pn{stmt} &\pp&
    \pn{controller{\textunderscore}stmt} \oo
    \pn{connection{\textunderscore}stmt} \oo
    \pn{assignment{\textunderscore}stmt} \\
  \pn{controller{\textunderscore}stmt} &\pp&
    \sm{controller} \ww
    \sm{(} \ww
    \pn{ident} \ww
    \sm{)} \\
  \pn{connection{\textunderscore}stmt} &\pp&
    \pn{param{\textunderscore}ref} \ww
    \sm{<-} \ww
    \pn{expr} \\
  \pn{assignment{\textunderscore}stmt} &\pp&
    \pn{ident} \ww
    \sm{=} \ww
    \pn{component{\textunderscore}stmt} \\
  \pn{component{\textunderscore}stmt} &\pp&
    \pn{ident} \ww
    \pn{lparen} \ww
    \pn{expr{\textunderscore}list} \ww
    \pn{rparen} \\
\end{eqnarray*}

Expressions
\begin{eqnarray*}
  \pn{expr} &\pp&
    \pn{constant} \oo
    \pn{param{\textunderscore}ref} \oo
    \pn{ident} \\
  \pn{expr{\textunderscore}list} &\pp&
    \pn{expr} \oo
    \pn{expr{\textunderscore}list} \ww
    \pn{comma{\textunderscore}sep} \ww
    \pn{expr} \\
  \pn{constant} &\pp&
    \pn{fconst} \oo
    \pn{iconst} \oo
    \pn{sconst} \\
  \pn{fconst} &\pp&
    \sd{floating point constant} \\
  \pn{iconst} &\pp&
    \sd{integer constant} \\
  \pn{sconst} &\pp&
    \sd{string constant} \\
  \pn{ident} &\pp&
    \sd{identifier} \\
  \pn{param{\textunderscore}ref} &\pp&
    \sm{in} \ww
    \sm{.} \ww
    \pn{ident} \oo
    \sm{out} \ww
    \sm{.} \ww
    \pn{ident} \oo \\
    & & \pn{ident} \ww
    \sm{.} \ww
    \sm{in} \ww
    \sm{.} \ww
    \sm{ident} \oo
    \pn{ident} \ww
    \sm{.} \ww
    \sm{out} \ww
    \sm{.} \ww
    \pn{ident} \oo \\
    & & \sm{(} \ww
    \pn{component{\textunderscore}stmt} \ww
    \sm{)} \ww
    \sm{.} \ww
    \sm{out} \ww
    \sm{.} \ww
    \pn{ident} \\
\end{eqnarray*}

Types and docstrings
\begin{eqnarray*}
  \pn{type} &\pp&
    \sm{file} \oo
    \sm{float} \oo
    \sm{int} \oo
    \sm{string} \oo
    \sm{[} \ww
    \pn{type} \ww
    \sm{]} \oo
    \pn{ident} \\
  \pn{docstring} &\pp&
    \sd{an optional documentation} \\
\end{eqnarray*}

Special productions
\begin{eqnarray*}
  \pn{cr} &\pp&
    \sd{A non-empty sequence of new lines} \\
  \pn{opt{\textunderscore}cr} &\pp&
    \sd{A sequence of new lines which may be empty} \\
  \pn{lbrace} &\pp&
    \sm{\{} \ww
    \pn{opt{\textunderscore}cr} \\
  \pn{rbrace} &\pp&
    \pn{opt{\textunderscore}cr} \ww
    \sm{\}} \\
  \pn{lbrace} &\pp&
    \sm{(} \ww
    \pn{opt{\textunderscore}cr} \\
  \pn{rbrace} &\pp&
    \pn{opt{\textunderscore}cr} \ww
    \sm{)} \\
  \pn{empty} &\pp&
    \nn \\
  \pn{comma{\textunderscore}sep} &\pp&
    \pn{opt{\textunderscore}cr} \ww
    \sm{,} \ww
    \pn{opt{\textunderscore}cr} \\
\end{eqnarray*}



\subsection{Type Checker}\label{sec:typechecker}


\subsection{XML generation}\label{sec:xml}
There is an implemented XML generator for an earlier version of the
implementation. Due to bigger changes other aspects were prioritized
and XML generation was left for the developers of Copernicus to
update. Since the language needs defenitions before they can be used
the XML generation is an iteration through thses defenitions. The only
non-straight forward part to implement is the instantiations of
atoms/networks as inputs in an assignment statement. The erlier
version of the XML generator replaced such input expressions with an
assignment statement and then connected inputs and outputs to
represent the same thing.

The XML generator is a visitor as the type checker. Visiting the
different nodes in an abstract syntax tree prints the corresponding
XML code and visit its children to generate the entire defenition with
indentations.

\subsection{Emacs mode}\label{sec:emacs}
The emacs mode only contains syntax highlighting. The mode was created
to get a more realistic view of the example code that were used during
designing the DSL and testing the implementation. The syntax
highlighting adds some understanding of what the code actually
represents which made it easier to add and change specific parts of
the DSL.

\section{Future work}


%\subsection{Language}
%The candidates for implementation language were: C, C++, C\#, F\#,
%Haskell, Java, OCamel. These candidates were discussed since the
%\emph{BNF converter} (BNFC) tool supports them. \citep{bnfc:online}
%Haskell was excluded first of all, mainly because of the restricted
%set of users. The developers of Copernicus needs to easily be able to
%modify the implementation when necessary. The next candidate was Java,
%but it was excluded because of the possible difficulties when
%installing java runtime environment on supercomputers. C was a good
%candidate since Copernicus is written in Python and C is easy to
%extend in Python, and most systems has a C compiler as default.
%
%The the implementation chosen was Python. As the rest of the system is
%written in Python, it is only natural to write extensions in the same
%language. UNIX systems have Python interpreter as default, which makes
%the extension easy to install aswell. The decision was also based on
%easy integration and maintenance against easy implementation. As BNFC
%is a powerfull tool, generating lexer, parser and abstract syntax tree
%from a simple BNF description, it is the easy implementation
%choice. Though this is important, easy integration and maintenance was
%prioritized.
%
%
%\subsection{Tools}
%BNFC was the first tool considered to be used. The main reason for
%this is that BNFC is a part of both \emph{Programming Languages} and
%\emph{Compiler Construction} which are courses tought at
%Chalmers. Experience and knowledge would have made it easy and
%effective to use. The reason why another tool was chosen is stated in
%the previous section.
%
%When the implementation language was established, the tools considered
%was PLY, YAPPS and SPARK\citep{ply:online,yapps:online,spark:online}.
%
%\highlight{SKRIV!!!}
%
%
%\subsection{Datastructures \& Methods}
%The abstract syntax tree is designed with a visitor pattern. The
%elements in the visitor pattern is build like a tree where each node
%has children. A node is a syntactic object where its children are
%values or other syntactic objects. By defining the visit methods for
%each sort of node, one perform operations based on a abstract syntax
%tree simply by visiting its top node. This way it is easy to write a
%type checker and XML generator.
%
%The type checker keeps environment information in a list of
%dictionaries, where each element in the list is a scope in the
%language. The dictionaries contain type information of functions and
%variables. New type definitions are stored in a seperate dictionary.

