\chapter{Codspeech}

\section{Project overview}


\section{Designing the DSL}


\section{Implementation}


\section{PLY (Python Lex-Yacc)}
PLY is an implementation of lex and yacc parsing tools, written purely
in python, by David Beazley \citep{ply:online}. It was originally
developed for an intrdoctory class on compilers back in 2001. It
provides most of the standard lex/yacc features including support for
empty productions, precedence rules, error recovery, and support for
ambigous grammars. It uses LR-parsing, which is a reasonabe parsing
scheme for larger grammars, but slightly restrics the type of grammars
that can be written (\highlight{citation needed}). PLY is
straightforward to use, and one its many advantages is the \emph{very}
extensive error checking, which certainly makes life easier.

\subsection{Python Lex}
The first step to implement the language is to write a tokenizer. This
is done with the Lex module of PLY. Language tokens are recognized
using regular expressions, and the steps are straighforward.

The names of all the token types are declard as a list of strings
named \texttt{tokens}.
\begin{lstlisting}
class CodspeechLexer(object):

...

    tokens = [
        # Literals: identifier, type, integer constant, float
        # constant, string constant
        'IDENT', 'ICONST', 'FCONST', 'SCONST', 'DOCSTRING',

        # Assignments: = :
        'EQUALS', 'COLON',

        # Connection: <-
        'CONNECTION',

        # Delimeters: ( ) { } [ ] , .
        'LPAREN', 'RPAREN',
        'LBRACE', 'RBRACE',
        'LBRACKET', 'RBRACKET',
        'COMMA', 'PERIOD',

        # Other:
        'CR', 'OPTIONAL', 'OPTIONS'
    ]
\end{lstlisting}

Tokens that require no special processing are declared using
module-level variables prefixed by \texttt{t_}, where the name
following \texttt{t_} has to exactly match some string in the tokens
list. Each such variable contains a regular expression string that
matches the respective token (Python raw strings are usually used
since they are the most convenient way to write regular expression
strings).
\begin{lstlisting}
class CodspeechLexer(object):

...

    t_EQUALS     = r'='
    t_COLON      = r':'
    t_CONNECTION = r'<-'
    t_LPAREN     = r'\('
    t_RPAREN     = r'\)'
    t_LBRACKET   = r'\['
    t_RBRACKET   = r'\]'
    t_LBRACE     = r'\{'
    t_RBRACE     = r'\}'
    t_COMMA      = r','
    t_PERIOD     = r'\.'
    t_OPTIONAL   = r'\?'
\end{lstlisting}

When tokens do require special processing, a token rule can be
specified as a function. For example, this rule matches numbers and
converts the string into a Python integer.
\begin{lstlisting}
    def t_ICONST(self, t):
        r'\d+'
        t.value = int(t.value)
        return t
\end{lstlisting}

In some cases, we may want to build tokens from more complex regular
expressions. For example:
\begin{lstlisting}
class CodspeechLexer(object):

...

    lowercase    = r'[a-z]'
    identchar    = r'[_A-Za-z0-9-]'
    ident        = r'(' + lowercase + r'(' + identchar + r')*)'

    def t_IDENT(self, t):
        # we want the docstring to be the identifier above
        ...
\end{lstlisting}

\noindent This is not possible to specify using a normal docstring. The
programmer would have to write the full RE, defeating the purpose of
re-usable code. However, there is a way around this by using the
\texttt{@TOKEN} decorator.
\begin{lstlisting}
from ply.lex import TOKEN

class CodspeechLexer(object):

...


    lowercase    = r'[a-z]'
    identchar    = r'[_A-Za-z0-9-]'
    ident        = r'(' + lowercase + r'(' + identchar + r')*)'

    @TOKEN(ident)
    def t_IDENT(self, t):
        t.type = self.keyword_map.get(t.value,"IDENT")
        return t
\end{lstlisting}

The observant reader might notice something special goind on in the
function \texttt{t_IDENT}. The processed string is checked against a
keyword map to decide wheter the token type should actually be
\texttt{IDENT} or something else. The keyword map is defined as a
dictionary, and the values are appended to the token list.
\begin{lstlisting}
class CodspeechLexer(object):

...

    keyword_map = {
        # Import
        'import'          : 'IMPORT',

        # Type
        'type'            : 'TYPE',

        # Atom keywords
        'atom'            : 'ATOM',
        #'options'         : 'OPTIONS',
        'python'          : 'ATOMTYPE',
        'python-extended' : 'ATOMTYPE',
        'external'        : 'ATOMTYPE',

        # Network
        'network'         : 'NETWORK',
        'controller'      : 'CONTROLLER',
                
        # Header
        'in'              : 'IN',
        'out'             : 'OUT',
        'default'         : 'DEFAULT',
        
        # Types
        'file'            : 'FILE',
        'float'           : 'FLOAT',
        'int'             : 'INT',
        'string'          : 'STRING',
    }


    tokens = [
        ...
    ] + list(set(keyword_map.values()))
\end{lstlisting}

\noindent Since our keyword map contains multiple keys mapping to the
same value and the token list can not contain any duplicates, the list
of values is converted to a set before it is converted back into a
list.


\subsection{Python Yacc}
The next implementation step was to write the parser using Python
Yacc. The list of tokens is imported from the tokenzer module and a
series of functions describing the grammar productions are defined.


\subsection{Alternative specification of Lexer and Parser}
As seen in the above examples, both the lexer and parser are defined
from instances of their own classes. The easiest way, however, is to
specify them directly in their own modules. The PLY documentation
explains this quite well, complete with examples. \citep{ply:online}

\section{Codspeech Backus-Naur Form}
\newcommand{\pn}[1]{\langle \textnormal{#1} \rangle}
\newcommand{\pp}{\models}
\newcommand{\oo}{\; \mid \;}
\newcommand{\sk}{\dots }
\newcommand{\ww}{\;}
\newcommand{\nn}{\perp}
\newcommand{\sm}[1]{\textnormal{#1}}
\newcommand{\sd}[1]{\textnormal{\it #1}}


Grammar production
\begin{eqnarray*}
  \pn{entrypoint} &\pp&
    \pn{opt{\textunderscore}cr} \ww \pn{program} \oo
    \pn{empty} \\
  \pn{program} &\pp&
    \pn{top{\textunderscore}def{\textunderscore}list} \ww
    \pn{opt{\textunderscore}cr} \\
  \pn{top{\textunderscore}def{\textunderscore}list} &\pp&
    \pn{top{\textunderscore}def} \oo
    \pn{top{\textunderscore}def{\textunderscore}list} \ww
    \pn{cr} \ww
    \pn{top{\textunderscore}def} \\
  \pn{top{\textunderscore}def} &\pp&
    \pn{import{\textunderscore}stmt} \oo
    \pn{newtype{\textunderscore}decl} \oo
    \pn{atom{\textunderscore}decl} \oo
    \pn{network{\textunderscore}decl} \\
\end{eqnarray*}

Import Statment
\begin{eqnarray*}
  \pn{import{\textunderscore}stmt} &\pp&
    \sm{import} \ww
    \pn{package{\textunderscore}path} \\
  \pn{package{\textunderscore}path} &\pp& 
    \pn{package{\textunderscore}identifier} \oo
    \pn{package{\textunderscore}path} \ww
    \sm{.} \ww
    \pn{package{\textunderscore}identifier} \\
  \pn{package{\textunderscore}identifier} &\pp& \pn{ident} \\
\end{eqnarray*}

Newtype
\begin{eqnarray*}
  \pn{newtype{\textunderscore}decl} &\pp&
    \sm{type} \ww
    \pn{type} \ww
    \pn{docstring} \ww
    \pn{cr} \ww
    \sm{(} \ww
    \pn{type{\textunderscore}decl{\textunderscore}list} \ww
    \sm{)} \\
  \pn{type{\textunderscore}decl{\textunderscore}list} &\pp&
    \pn{type{\textunderscore}decl} \oo
    \pn{type{\textunderscore}decl{\textunderscore}list} \ww
    \pn{comma{\textunderscore}sep} \ww
    \pn{type{\textunderscore}decl} \\
  \pn{type{\textunderscore}decl} &\pp&
    \pn{type} \ww
    \sm{:} \ww
    \pn{ident} \\
\end{eqnarray*}

Atom Declaration
\begin{eqnarray*}
  \pn{atom{\textunderscore}decl} &\pp&
    \sm{atom} \ww
    \pn{atomtype} \ww
    \pn{header} \ww
    \pn{optionblock} \\
  \pn{atomtype} &\pp&
    \sm{python} \oo
    \sm{python-extended} \oo
    \sm{external} \oo \\
  \pn{optionblock} &\pp&
    \sm{option} \ww
    \pn{opt{\textunderscore}cr} \ww
    \sm{(} \ww
    \pn{atom{\textunderscore}optio{\textunderscore}list} \ww
    \sm{)} \\
  \pn{atom{\textunderscore}option{\textunderscore}list} &\pp&
    \pn{atom{\textunderscore}option} \oo
    \pn{atom{\textunderscore}option{\textunderscore}list} \ww
    \pn{comma{\textunderscore}sep} \ww
    \pn{atom{\textunderscore}option} \\
  \pn{atom{\textunderscore}option} &\pp&
    \pn{option{\textunderscore}ident} \ww
    \sm{:} \ww
    \pn{option{\textunderscore}ident} \ww\\
  \pn{option{\textunderscore}ident} &\pp&
    \pn{ident} \oo
    \pn{option{\textunderscore}ident} \ww
    \sm{.} \ww
    \pn{ident} \\
\end{eqnarray*}

Network Declaration
\begin{eqnarray*}
  \pn{network{\textunderscore}decl} &\pp&
    \sm{network} \ww
    \pn{header} \ww
    \pn{networkblock} \\
  \pn{networkblock} &\pp&
    \pn{stmt{\textunderscore}block} \\
\end{eqnarray*}

Atom and Network Header
\begin{eqnarray*}
  \pn{header} &\pp&
    \pn{ident} \ww
    \pn{docstring} \ww
    \pn{cr} \ww
    \pn{inputs} \ww
    \pn{outputs} \\
  \pn{inputs} &\pp&
    \sm{in} \ww
    \sm{(} \ww
    \pn{input{\textunderscore}list} \ww
    \sm{)} \ww
    \pn{cr} \oo
    \sm{in} \ww
    \pn{no{\textunderscore}params} \ww
    \pn{cr} \\
  \pn{outputs} &\pp&
    \sm{out} \ww
    \sm{(} \ww
    \pn{output{\textunderscore}list} \ww
    \sm{)} \ww
    \pn{cr} \oo
    \sm{out} \ww
    \pn{no{\textunderscore}params} \ww
    \pn{cr} \\
  \pn{input{\textunderscore}list} &\pp&
    \pn{input} \oo
    \pn{input{\textunderscore}list} \ww
    \pn{comma{\textunderscore}sep} \ww
    \pn{input} \\
  \pn{output{\textunderscore}list} &\pp&
    \pn{output} \oo
    \pn{output{\textunderscore}list} \ww
    \pn{comma{\textunderscore}sep} \ww
    \pn{output} \\
  \pn{no{\textunderscore}params} &\pp&
    \sm{(} \ww
    \pn{opt{\textunderscore}cr} \ww
    \sm{)} \\
  \pn{input} &\pp&
    \pn{type} \ww
    \pn{ident} \ww
    \pn{docstring} \oo
    \sm{?} \ww
    \pn{type} \ww
    \pn{ident} \ww
    \pn{docstring} \oo \\
    & & \pn{type} \ww
    \pn{ident} \ww
    \sm{default} \ww
    \pn{constant} \ww
    \pn{docstring} \\
  \pn{output} &\pp&
    \pn{type} \ww
    \pn{ident} \ww
    \pn{docstring} \oo \\
    & & \pn{type} \ww
    \pn{ident} \ww
    \sm{default} \ww
    \pn{constant} \ww
    \pn{docstring} \\
\end{eqnarray*}

Statements
\begin{eqnarray*}
  \pn{stmt{\textunderscore}block} &\pp&
    \pn{opt{\textunderscore}cr} \ww
    \pn{lbrace} \ww
    \pn{stmt{\textunderscore}list} \ww
    \pn{rbrace} \oo \\
    & & \pn{opt{\textunderscore}cr} \ww
    \pn{stmt{\textunderscore}block{\textunderscore}empty} \\
  \pn{stmt{\textunderscore}block{\textunderscore}empty} &\pp&
    \sm{(} \ww
    \pn{opt{\textunderscore}cr} \ww
    \sm{)} \\
  \pn{stmt{\textunderscore}list} &\pp&
    \pn{stmt} \oo
    \pn{stmt{\textunderscore}list} \ww
    \pn{cr} \ww
    \pn{stmt} \\
  \pn{stmt} &\pp&
    \pn{controller{\textunderscore}stmt} \oo
    \pn{connection{\textunderscore}stmt} \oo
    \pn{assignment{\textunderscore}stmt} \\
  \pn{controller{\textunderscore}stmt} &\pp&
    \sm{controller} \ww
    \sm{(} \ww
    \pn{ident} \ww
    \sm{)} \\
  \pn{connection{\textunderscore}stmt} &\pp&
    \pn{param{\textunderscore}ref} \ww
    \sm{<-} \ww
    \pn{expr} \\
  \pn{assignment{\textunderscore}stmt} &\pp&
    \pn{ident} \ww
    \sm{=} \ww
    \pn{component{\textunderscore}stmt} \\
  \pn{component{\textunderscore}stmt} &\pp&
    \pn{ident} \ww
    \pn{lparen} \ww
    \pn{expr{\textunderscore}list} \ww
    \pn{rparen} \\
\end{eqnarray*}

Expressions
\begin{eqnarray*}
  \pn{expr} &\pp&
    \pn{constant} \oo
    \pn{param{\textunderscore}ref} \oo
    \pn{ident} \\
  \pn{expr{\textunderscore}list} &\pp&
    \pn{expr} \oo
    \pn{expr{\textunderscore}list} \ww
    \pn{comma{\textunderscore}sep} \ww
    \pn{expr} \\
  \pn{constant} &\pp&
    \pn{fconst} \oo
    \pn{iconst} \oo
    \pn{sconst} \\
  \pn{fconst} &\pp&
    \sd{floating point constant} \\
  \pn{iconst} &\pp&
    \sd{integer constant} \\
  \pn{sconst} &\pp&
    \sd{string constant} \\
  \pn{ident} &\pp&
    \sd{identifier} \\
  \pn{param{\textunderscore}ref} &\pp&
    \sm{in} \ww
    \sm{.} \ww
    \pn{ident} \oo
    \sm{out} \ww
    \sm{.} \ww
    \pn{ident} \oo \\
    & & \pn{ident} \ww
    \sm{.} \ww
    \sm{in} \ww
    \sm{.} \ww
    \sm{ident} \oo
    \pn{ident} \ww
    \sm{.} \ww
    \sm{out} \ww
    \sm{.} \ww
    \pn{ident} \oo \\
    & & \sm{(} \ww
    \pn{component{\textunderscore}stmt} \ww
    \sm{)} \ww
    \sm{.} \ww
    \sm{out} \ww
    \sm{.} \ww
    \pn{ident} \\
\end{eqnarray*}

Types and docstrings
\begin{eqnarray*}
  \pn{type} &\pp&
    \sm{file} \oo
    \sm{float} \oo
    \sm{int} \oo
    \sm{string} \oo
    \sm{[} \ww
    \pn{type} \ww
    \sm{]} \oo
    \pn{ident} \\
  \pn{docstring} &\pp&
    \sd{an optional documentation} \\
\end{eqnarray*}

Special productions
\begin{eqnarray*}
  \pn{cr} &\pp&
    \sd{A non-empty sequence of new lines} \\
  \pn{opt{\textunderscore}cr} &\pp&
    \sd{A sequence of new lines which may be empty} \\
  \pn{lbrace} &\pp&
    \sm{\{} \ww
    \pn{opt{\textunderscore}cr} \\
  \pn{rbrace} &\pp&
    \pn{opt{\textunderscore}cr} \ww
    \sm{\}} \\
  \pn{lbrace} &\pp&
    \sm{(} \ww
    \pn{opt{\textunderscore}cr} \\
  \pn{rbrace} &\pp&
    \pn{opt{\textunderscore}cr} \ww
    \sm{)} \\
  \pn{empty} &\pp&
    \nn \\
  \pn{comma{\textunderscore}sep} &\pp&
    \pn{opt{\textunderscore}cr} \ww
    \sm{,} \ww
    \pn{opt{\textunderscore}cr} \\
\end{eqnarray*}


\subsection{Abstract Syntax Tree}
\subsection{Parser}
\subsection{Typechecker}
\subsection{XML generation}
\subsection{emacs mode}


\section{Future work}
