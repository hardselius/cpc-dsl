\documentclass[a4paper]{article}
\usepackage[utf8]{inputenc}

\usepackage{tikz}
\usepackage{gantt}

\title{Planning report\\
  \large{Domain-specific language for high-level
  sampling tasks in high-performance computing
  }
}

\author{Martin Hardselius \and Viktor Almqvist} 

\date

\begin{document}

\maketitle
\newpage

\section{Background}
\begin{quotation}
  ``Many applications for high-performance computing, including
  bio-molecular simulations and materials science calculations, have
  limited potential to achieve the strong scaling necessary to run on
  modern supercomputers with 10000 cores or more. At the same time,
  many applications of these simulations are statistical in nature:
  they rely on sampling of many individual simulations or
  trajectories\ldots

  \ldots In response to this, we have created a framework for the
  execution of large-scale sampling calculations: Copernicus. This
  platform changes the focus away from managing individual simulations
  and processing the results and expresses large multi-level jobs as
  single tasks. The user can create and compose these computational
  tasks, and inspect a task's progress while it is running.''
\end{quotation}

Copernicus is a framework for managing large scale calculations and
the distribution of these computational tasks over a large scale of
workers. The system contains servers, workers and clients. The server
distributes the tasks to workers and the client sends calculation
requests to the server. The calculations, which are the input to
Copernicus, are described using a dataflow-network model. The
data-flow networks are basically instances of functions and their
connections to each other. The data-flow networks are currently
defined in the form of XML code. Since the developers of Copernicus
aims for a more user-friendly experience, they want to replace the XML
input with a domain-specific language.

\section{Aim}
The goal of this project is the creation of a domain-specific language
for describing input to the project Copernicus. Since the input to
Copernicus can be described as data-flow network, the design of this
DSL should adhere that paradigm. Furthermore, the language should
allow Copernicus plug-in libraries, and the internal network must be
algorithmically expressible, to allow the user to query a full
state. The language should be simple enough to describe data-flow
networks as plain code, but also designed with the possibility to
easily add a graphical interface.

\section{Problem}
The objective is to implement a domain specific language as described
in the previous section. The intended users of the language are not
necessarily programmers, which makes it important for the language to be
simple and intuitive.

The DSL should be fully functional which mean it must include all
features and structures offered in the data-flow networks in
Copernicus. It should be possible to pipe the compiler to Copernicus,
which means the output form the compiler should be in the same form as
the input for Copernicus.


\section{Scope}
The most important part is to get a working solution since the DSL is
supposed to be used by the users of Copernicus. The input to the
system is currently described in XML, but the DSL and its compiler
could render the use of XML redundant. As the main objective is a
fully functional compiler, we will generate XML code. This saves time
and will enable us to focus on the more important parts of our
project. If there is time when we are done with the translation from
our language to XML, we will look into the possibility of an extension
that bypasses the XML-code generation.

\section{Method}
Our method for making the DSL is divided into three steps: research,
design and implementation. We will have short daily meetings with our
supervisors during the first period of the research step, where we
will discuss what has been done since the last meeting and what we
will be doing until the next one. We will meet less often later on,
since the first part is to make sure we get started and on the right
track.

\subsection{Research}
Our first objective is to research languages describing data-flow
networks to get a good sense on how such languages are designed. This
will serve as an inspiration to our DSL design.

The next objective is to learn the structure of the current input
data. We will be studying project code, the XML-code examples used for
testing the system and discuss the DSL design with the Copernicus
developers. We need to understand how the input data hierarchy works,
how functions and sub nets are defined, etc.

\subsection{Design}
In the end of the research period, we will start to define a language
syntax. We expect to make a simple draft to show our ideas for a
structure of the language. During syntax development, we will discuss
our draft regularly with the Copernicus developers to avoid missing
any important features.

Working from this point with a continuous input from the developers
will result in us creating a good syntax that both the developers and
we are satisfied with. The syntax need to be clean and be clear and
effective to implement.

\subsection{Implementation}
\subsubsection{Parser}
We can start implementing the compiler when the language syntax has
been established. The first step is to write a parser for building
abstract syntax trees. We will use BNFC which is a tool for generating
a finished parser from BNF grammar. Our objective is to write a BNF
grammar which represent our syntax defined in the previous step. The
grammar will contain definitions of statements, expressions, types,
etc. The reasons behind choosing BNFC are former experience using it
and the freedom to decide which implementation language to use later
on; BNFC has support for C, C++, C\#, F\#, Haskell, Java, OCamel.

We have chosen not to write the compiler in Haskell, a language we
have a lot of experience in using. There are two reasons for this
decision: The first is to ensure that anyone can more easily take over
and understand our code (or write a graphical interface) when we leave
the project, since Java, C, C++, C\# are more commonly known
languages. The second reason is that we want to strengthen our skills
in some other language than Haskell before we graduate from Chalmers.

\subsubsection{Type Checker}
Since the language is supposed to be typed, a type-checker will be
implemented. A type-checker can provide the user with more expressive
error reporting, and will also facilitate the learning experience. At
this point, we will have decided on an implementation language.

\subsubsection{Code Generation}
Once we have a type checker we can start with the final part of the
compiler, the code generator. The code generator will generate XML
code which can be used directly as input for Copernicus. It is crucial
during this part to fully understand the structure of the XML code
describing data-flow networks. At this point we can start trying out
our project with Copernicus as we add parts of the code generator.


\section{Time Plan}
\begin{gantt}[
    xunitlength=0.5cm,
    fontsize=\small,
    titlefontsize=\small,
    drawledgerline=true]
  {8}{18}
  \begin{ganttitle}
    \titleelement{weeks}{18}
  \end{ganttitle}
  \begin{ganttitle}
    \numtitle{7}{1}{24}{1}
  \end{ganttitle}
  \ganttbar{research}{0}{4}
  \ganttbar{design}{2}{4}
  \ganttbar[pattern=crosshatch,color=red]{easter}{7}{1}
  \ganttbar{implementation}{5}{8}
  \ganttbar{report}{11}{5}
  \ganttbar{presentation}{16}{2}
\end{gantt}

\end{document}
